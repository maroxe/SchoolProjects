
%%% Local Variables:
%%% mode: latex
%%% TeX-master: "main"
%%% End:

\section{Introduction}

J'ai fait mon stage a JP Morgan equipe exotic rates.
La plus grosse partie du travail: trouver le bon modele pour pricer un produit et le calibrer au marche

L'interêt que porte les banques aux produits exotiques a considérablement augmenté aucours des dernières années. Plusieurs modèles ont été dévelopé pour refeléter au mieux le comportement des marchés financiers.
Le département au sein duquel j'ai effectué mon stage s'occupait des produits basé sur les taux d'intêret. Dans ce domaine, une propriété appréciable est le fait qu'ils possèdent un nombre de paramètres suffisant pour être calibrés parfaitement au prix observable dans la réalité, sans pour autant ``overfitter'' l'échantillon disponible. En effet, depuis la crise, le marché des taux a connu des changement radicaux, la possiblité d'observer des taux négatifs en est un example. 

Un produit en particulier n'a cessé de  gagner en notoriété: les cancelables spread options. Ma mission de stage était de comprendre les modèles existant et leurs implémentations, comprendre leurs limites, et essayer de trouver des améliorations.


Dans la littérature, il existe differents modoèes Hull white modele 1F et 2F 
Motivation
C'est la premiere implementation de de WH a 2F avec des parametres variant avec le temps


Dans une première partie je presenterai quelques définition essentielles à l'études des taux d'interêts, ainsi que les produits financiers principaux utilisé pour la calibration des modèles.
Dans la deuxième partie modèle existant dans la littérature, pk 2F?
Implémentation, calibration
Conclusion

