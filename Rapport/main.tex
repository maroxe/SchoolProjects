%%% Local Variables:
%%% mode: latex
%%% TeX-command-extra-options: "-shell-escape"
%%% TeX-master: t
%%% End:

%%% Preamble
\documentclass[paper=a4, fontsize=11pt]{scrartcl}
\usepackage[utf8]{inputenc}  
\usepackage[T1]{fontenc}
\usepackage{fourier}
\usepackage[english]{babel}															% English language/hyphenation
\usepackage[protrusion=true,expansion=true]{microtype}	
\usepackage{amsmath,amsfonts,amsthm} % Math packages
\usepackage[pdftex]{graphicx}	
\usepackage{url}
\usepackage{tikz}
%\usepackage{minted}

%%% Custom sectioning
\usepackage{sectsty}
\allsectionsfont{\centering \normalfont\scshape}

%%% Custom headers/footers (fancyhdr package)
\usepackage{fancyhdr}
\pagestyle{fancyplain}
\fancyhead{}											% No page header
\fancyfoot[L]{}											% Empty 
\fancyfoot[C]{}											% Empty
\fancyfoot[R]{\thepage}									% Pagenumbering
\renewcommand{\headrulewidth}{0pt}			% Remove header underlines
\renewcommand{\footrulewidth}{0pt}				% Remove footer underlines
\setlength{\headheight}{13.6pt}


%%% Equation and float numbering
\numberwithin{equation}{section}		% Equationnumbering: section.eq#
\numberwithin{figure}{section}			% Figurenumbering: section.fig#
\numberwithin{table}{section}				% Tablenumbering: section.tab#


%%% Maketitle metadata
\newcommand{\horrule}[1]{\rule{\linewidth}{#1}} 	% Horizontal rule
\newcommand{\espr}[1]{
  \mathrm{E}^Q \left[ #1 \right]
}

\newcommand{\Qespr}[2]{
  \mathrm{E}^{#1} \left[ #2 \right]
}


\title{
		%\vspace{-1in} 	
		\usefont{OT1}{bch}{b}{n}
		\normalfont \normalsize \textsc{Ecole Polytechnique} \\ [25pt]
		\horrule{0.5pt} \\[0.4cm]
		\huge Modèles des Taux d'interêt \\
		\horrule{2pt} \\[0.5cm]
}
\author{
		\normalfont \normalsize
                Bachir EL KHADIR\\[-3pt] \normalsize
                \today	
}
\date{}

\theoremstyle{definition}
\newtheorem{theorem}{Theorem}
\newtheorem{defn}{Definition}

\newcommand{\IMG}[3]{
  \begin{center}
    \includegraphics[scale=#3]{#1}%
    \end{center}
}


%%% Begin document
\begin{document}
\maketitle

\newpage
\tableofcontents

\newpage

%%% Local Variables:
%%% mode: latex
%%% TeX-master: "main.tex"
%%% End:

\chapter*{Remerciments}

Je tiens à remercier toutes les personnes qui ont contribué au succès de mon stage et qui m'ont aidé lors de la rédaction de ce rapport.

Tout d'abord, j'adresse mes remerciements à mon professeur, Mr Stefano De Marco de l'Ecole Polytechnique qui m'a beaucoup aidé dans ma recherche de stage et ma permis de postuler dans cette entreprise. Son écoute et ses conseils m'ont permis de cibler mes candidatures et de trouver ce stage me correspondait totalement.

Je tiens à remercier vivement mon maitre de stage, Mr John Paul Barjaktarevic, responsable du service Y au sein de la banque JP Morgan, pour son accueil, le temps passé ensemble et le partage de son expertise au quotidien. Grâce aussi à sa confiance j'ai pu m'accomplir totalement dans mes missions avec son aide précieuse dans les moments les plus délicats.

Je remercie également toute l'équipe ``rates exotic'' pour leur accueil, leur esprit d'équipe. 

Enfin, je tiens à remercier toutes les personnes qui m'ont conseillé et relu lors de la rédaction de ce rapport de stage : ma famille, mon amie Julie B camarade de promotion.


\newpage

%%% Local Variables:
%%% mode: latex
%%% TeX-master: "main"
%%% End:

\newpage

\subsection*{Résumé}
  
  Ce rapport présente quelques outils utilisés lors du pricing de dérivés de taux. 
  Il est particulièrement centré sur le modèle de Hull White qui est conduit par deux facteurs (appelé aussi G2++).  J'adopterai le plan suivant dans la rédaction de ce rapport:
  \begin{itemize}
  \item 
    La première partie introduit le marché de taux par un certain nombre de définitions qui lui sont propores, ainsi que les produits financiers principaux utilisés pour la calibration des modèles.
  \item 
    Dans la deuxième partie je présenterai quelques modèles existant, et je discuterai de leurs limites et de leurs implémentation pratique.
  
  \item
    La troisième partie présente les résultats concrets que j'ai obtenus dans le cas du modèle gaussien à deux facteurs, ainsi que la partie calibratoin de modèle.
  \end{itemize}
  
\subsection*{Abstract} 
This report presents some tools used in the pricing for intereset rates derivatives. It's focused on two factors Hull-White model (called G2++).  
  \begin{itemize}
  \item
    The first part introduces some definitions relative to the interest rates market, as well as some derivative used later for the calibration process.
  \item
    In the second part I will present some existing models, their limitations and possible implementation.

  \item
    The last part presents the results I obtained using the G2++ model and the calibration process.
  \end{itemize}




\newpage
\chapter*{Introduction}

J'ai fait mon stage a JP Morgan equipe exotic rates. La plus grosse partie du travail consitait à trouver le bon modèle mathématique pour pricer un produit et le calibrer au marche.

L'interêt que porte les banques aux produits exotiques a considérablement augmenté aucours des dernières années. Plusieurs modèles ont été dévelopé pour refeléter au mieux le comportement des marchés financiers.

Le département au sein duquel j'ai effectué mon stage s'occupait des produits basé sur les taux d'intêret. Dans ce domaine, une propriété appréciable dans un modèle est le fait qu'il possèd un nombre de paramètres suffisant pour être calibré parfaitement aux prix observable dans la réalité, sans pour autant ``overfitter'' l'échantillon disponible. En effet, dérnièrement, surtout depuis la crise, le marché des taux a connu des changement radicaux, la possiblité d'observer des taux négatifs en est un example. 

Un produit en particulier n'a cessé de  gagner en notoriété: les cancelables spread options. Ma mission de stage était de comprendre les modèles existant et leurs implémentations qui permettent de pricer (entre autres) ce type de produits, comprendre leurs limites, et essayer de trouver des améliorations.

Ce stage est à forte composante informatique, une attention particulière a été accordée aux détails d'implémentation et optimisation du code.
 
\newpage
%%% Local Variables:
%%% mode: latex
%%% TeX-master: "main"
%%% End:


\chapter{Préliminaires sur les taux d'intérêt}

\section{Définition}
\begin{defn}
On dénote le prix de zéro d'une obligation zéro-coupon $P(T, S)$ le montant qu'il faut investir à dans un instrument risque-neutre au temps $T$ pour obtenir une unité de monnaie au temps $S$.
\end{defn}

\begin{defn}
On définit $f(t, T)$ le taux d'intérêt instantané forwad à la date $t$ pour une maturité $T$ la quantité $$f(t, T) := - \frac{ \delta}{\delta T}  log P(T, S)$$
\end{defn}

\begin{defn} Le taux instantané est défini par
  $r_t = \underset{T \to t}{lim}f(t, T) $ \\
  Le taux d'actualisation (stochastique) est: $D_t^T := e^{-\int_t^Tr_s \rm{d}s}$
\end{defn}

\iffalse
\begin{defn}
  Le taux d'intérêt cumulé entre deux période $t$ et $T$ est la quantité $R(t, T)$ que $r_t$ doit égaler pour avoir le même rendement
\end{defn}
\fi

\begin{defn}
  Le taux d'intérêt forward $F(t; T, S)$ est la prévision  à l'instant $t$ du taux entre deux période $T$ et $S$.
  Ce taux est la quantité $L$ connu à l'instant $t$ qui annule la valeur du contrat suivant à l'instant $t$:
  \begin{itemize}
  \item Recevoir l'intérêt  $L$ sur un 1 euro entre $T$ et $S$
  \item Payer le taux variable  $F(T, S)$ sur un 1 euro entre $T$ et $S$
  \end{itemize}
\end{defn}


le taux $r_t$ n’est pas un produit échangé sur le marché que l’on peut mettre en portefeuille. On ne peut donc pas construire de couverture d’un produit donné de la même manière que dans un modèle d’action, et ce malgré la similitude des modèles mathématiques.

\section{Mesures équivalentes}

Nous nous plaçons dans le cadre d'une économie à temps continu, qui admet un espace de probabilité $(\Omega, \cal{F}, \mathbb{P})$, avec $K+1$ actifs tradables, que nous appellerons actifs de base, dont le prix est donné par $(S_t = (S^0_t, ...S^k_t))_{t \geq 0}$. Dans toute la suite nous confondons l'actif et son prix.
$S^0$ étant l'actif sans risque, qui évolue donc au temps sans risque $$\mathrm{d}S^0_t = r_t S^0_t \mathrm{d}t$$
ie $$S^0_t = e^{\int_0^t r_s \mathrm{d}s}$$

Par définition, nous connaissons le prix des actifs $K+1$, dans la prochaine section nous détaillerons la procédure de pricing de produits plus compliqués.

\subsubsection{Principe de pricing}
A travers les $K+1$ actifs de base, nous construisons des produits plus complexes. 
Le prix d'un tel produit est donc intimement lié à la possibilité de trouver une stratégie autofinancée qui le réplique.
Commençons d'abord par définir ce qu'est une stratégie auto financé.

\begin{defn}
  \begin{itemize}
  \item Une stratégie est un processus $(\Phi_t = (\Phi^0_t, ... \Phi^K_t))_t$ localement borné et adapté à la filtration $\cal{F}$.
  \item La valeur associé à cette stratégie est donné par $V_t(\Phi) = <\Phi_t, S_t>$.
  \item Une stratégie est auto financée sir $\mathrm{d}V_t = \Phi_t \mathrm{d}S_t$
  \end{itemize}
\end{defn}

Une hypothèse souvent utilisée dans le cadre de la finance de marché est l'absence d'arbitrage. Une opportunité d'arbitrage est la possibilité d'investir 0 aujourd'hui, et recevoir, avec probabilité non nulle, un montant positive dans le future. En d'autres termes, l'absence d'arbitrage signifie que si $\Phi$ est une stratégie auto financée telle que $V_0(\Phi) = 0$, alors $\mathbb{P} ( V_t(\Phi) > 0 ) = 0$. Ceci nous permettra de valoriser des produits complexes en répliquant leur payoff par une combinaison linéaire de produits simples dont le prix est connue.

Une deuxième hypothèse que nous admettrons dans la suite est la complétude du marché: Tout les produits utilisés seront considérés disponibles à tout moment et en quantité abondante (liquide), ie à chaque instant $t$, pour tout payoff $H$, il existe une stratégie autofinancée associée $\Phi$ qui vérifie $V_t( \Phi ) = H$. Nous ne traiterons pas le cas des produits illiquides. Ceci est justifié, le marché des taux étant l'un des plus gros en volume dans le monde.

Nous pouvons montrer( \cite{Brugo}) que ces hypothèse sont équivalentes à l'existence d'une mesure de probabilité risque neutre $Q$ unique sous laquelle le prix actualisé de tous les produits tradables sont des martingales. ie si un on note $H_t$ le prix à l'instant $t$ d'un produit financier, alors
$$H_t =\espr{ \frac{H_s}{S^0_t} | F_t } = V_t(\Phi)$$

En particulier, le prix d'un zéro coupon qui paye 1 à l'instant $T$ est donné par
$$ P(t, T) := \espr{  e^{-\int_t^T r} } $$


Nous pouvons interpréter le ratio $\frac{H_s}{ S^0_s}$ comme étant le nombre  de $H$ par unité de facteur d'actualisation stochastique $S^0$. Le facteur d'actualisation est appelé dans ce cas numéraire. Nous verrons maintenant que nous pouvons choisir un autre numéraire plus adapté au produit que nous voulons pricer, puisque le changement de numéraire préserve la propriété d'autofinancement d'un portefeuille. \cite{Hull}

\begin{defn} Un numéraire est tout actif financier ne payant pas de dividendes \end{defn}

\begin{defn} Mesure de probabilité équivalente.
  
Supposons qu’il existe un numéraire $(M_t )_{t \geq 0}$ et une mesure martingale équivalente $Q^M$ telle que le prix de chaque actif actualisé par le processus M soit une$Q_M$-martingale. 
$$  (\forall i) \frac{S^i_t}{M_t} = \Qespr{Q^M}{ \frac{S^i_T}{M_T} | F_t}$$
Soit $(N_t )_{t \geq 0}$ un numéraire. \\
Alors il existe une mesure de probabilité $Q_N$ telle que le prix de chaque actif actualisé par le processus $N$ soit une $Q_N$-martingale, ie.
$$ (\forall i) \frac{S^i_t}{N_t} = (\forall i) \Qespr{Q^N}{ \frac{S^i_T}{N_T} | F_t}$$

où $Q^N$ est définie par:
$$\Qespr{Q^N}{ H } = \Qespr{Q^M}{ \frac{ M_T/N_T}{M_0/N_0} H}$$

\end{defn}

\textbf{Exemple:} Mesure forward neutre

Le bond zéro coupon dont la maturité coïncide avec la date du payement d'un produit financier peut servir de numéraire. Nous appellerons la mesure de probabilité associé $Q_T$.

Dans ce cas $P(T, T) = 1$, et par conséquent il suffit de calculer l'espérance du payoff (divisé par 1) sous $Q_T$.
Si nous notons le payoff de ce produit $H$, alors son prix à l'instant $0$ est donné par $$P(t, T) \, \Qespr{Q_T}{ H | F_t } $$
Pour que cela nous soit utile, il faut que la dynamique de $H$ soit connue sous $Q_T$. Ceci est vérifié pour les contrats payant un taux d'intérêts sur un nominal fixe. En effet $(F(t; S, T))_t$ est une martingale 
$$ \Qespr{Q_T}{ F(t; S, T) | F_u } = F(u; S, T)$$

\textbf{Preuve:}
Si nous disposons de  $\frac{P(t, S)}{1+(T-S)F(t; S, T)}$ au temps $t$, nous pouvons acheter $\frac{1}{1+(T-S)F(t; S, T)}$ unités de l'obligation $P(t, S)$, nous obtenons $\frac{1}{1+(T-S)F(t; S, T)}$ au temps $S$, cette somme là est, par définition de $F$, équivalente à l'obtention de 1 à l'instant $T$, qui exactement le payoff de l'obligation $P(t, T)$.
Par principe de \textbf{non arbitrage}, ces deux investissement doivent avoir le même coût, ie: $$\frac{P(t, S)}{1+(T-S)F(t;S,T)} = P(t, T)$$, ou encore
$$ \frac{1}{T-S} \left( \frac{P(t, S)}{P(t, T)} - 1  \right) $$
La preuve en découle.

\newpage

\section{Produits financier d'intérêt}

Le développement de la section précédente nous sera utile pour pricer les dériver des taux.
Considérons le cas particulier d'un call européen à maturité $T$, strike $K$, dont le sous-jacent est bond zéro coupon qui expire à l'instant $S$. Le payoff d'un tel contrat est connu: $ (P(T, S) - K)^+)$. Son prix à un instant antérieur $t$ est
$$ZBC(t, T, S, K) := \espr{ e^{-\int_t^T r_s \rm{d}s} \, (P(T, S) - K)^+ | F_t }$$
Il est plus pratique de considérer la forward mesure, sous laquelle le prix du call s'écrit
$$ZBC(t, T, S, K) = P(t, T) \, \Qespr{Q_T}{(P(T, S) - K)^+ | F_t}$$
De même, pour un put
$$ZBP(t, T, S, K) = P(t, T) \, \Qespr{Q_T}{(K - P(T, S))^+ | F_t}$$

Cette écriture nous rappelle la formule de blackscholes pour les options sur les actions.

\begin{defn}
  Swap:
Un swap est un contrat entre deux parties qui s'engagent à échanger des flux financiers pendant une durée et à une fréquence déterminées. La plupart du temps, ces flux sont déterminé comme étant l'intérêt à un taux fixe $K$ contre un taux variable (taux Libor ${L(T_i)}_i$ par exemple) sur un notionnel $N$. 
$$ N \sum D_t^{T_i} \tau_i (L(T_{i-1}) - K) $$
\end{defn}


\begin{defn}
  Caplet:
  Un caplet peut être vu comme un call/put européenne sur un
  Son payoff est le suivant
$$ \tau (L(T, S) - K)^+ $$
\end{defn}

\begin{align}
  Cpl(t, T, S, \tau, X)
  &= \espr{ e^{-\int_t^S r_s \rm{d}s} \tau (L(T, S) - K)^+ | F_t} \\ 
  &= \espr{ e^{-\int_T^S r_s \rm{d}s} P(t, T)  \tau (L(T, S) - K)^+ | F_t} \\
  &= \espr{ e^{-\int_T^S r_s \rm{d}s} (1 - (1 + X \tau)P(t, T))^+ | F_t} \\
  &= (1 + X \tau) \espr{ e^{-\int_T^S r_s \rm{d}s} (\frac{1}{1 + X \tau} - P(t, T))^+ | F_t} \\
  &= (1+X \tau) ZBP(t, T, S, \frac{1}{1+X \tau})
\end{align}

\begin{defn}
  Cap:
Un cap peut être vu comme une somme de caplets
$$ N \sum D_t^{T_i} \tau_i (L(T_{i-1}) - K)^+ $$
\end{defn}

La forme des payoff indique que le cap permet de protéger son détenteur d’une hausse des taux Libor, et symétriquement que le floor protège d’une éventuelle baisse de ces taux.

\begin{defn}
  Swaption:

 Une swaption payeuse européenne est une option permettant d’entrer, à une date $T$ appelé maturité, dans un swap payeur pour la période $(\alpha, \beta)$ de nominal $N$ et de strike $K$. 
 Les pâment s'effectuent aux instants $T_{\alpha} \leq T_i \leq T_{\beta}$. A la date $T_\alpha$, la valeur du swap payeur sous-jacent s'écrit:
$$ N \sum_{i} P(T_{\alpha}, T_i) (T_{i+1} - T_i) \left(L(T_{\alpha}, T_i) - K \right)$$
La valeur de la swaption s'écrit alors:
$$ N \left[ \sum_{i} P(T_{\alpha}, T_i) (T_{i+1} - T_i) \left(L(T_{\alpha}, T_i) - K \right) \right]^+$$
  
\end{defn}

Il est intéressant de noter que l'expression d'un swaption en fonction des taux sous-jacents n'est plus linéaire comme dans le cas des caps. 

%%% Local Variables:
%%% mode: latex
%%% TeX-master: "main"
%%% End:
 
\newpage
\section{Les différents modèles des taux d'interêts}

Nous avons vu dans la partie précédente que pour trouver un prix aux dérivés de taux, il faut donner la dynamique qui régit le sous-jacent, dans notre cas  *BLA BLA BLA*
Plusieurs approches sont possibles. Nous pouvons modéliser directement le forward:

$$\mathrm{d} L = \sigma L^\beta \mathrm{d}W_t$$

L'aproche historique, décrit la dynamique du taux d'interêt instantané comme étant ``drivé'' par un driver à une seule dimension. C'est l'approche que nous adopterons ici.
Ceci est pratique dans le seul où les prix de zéro coupon et le taux sont directement disponible dans le modèle.
De plus, certains produits financiers dépendent directement la courbe de rendement.
Nous avons vu précédemment que la donnée du taux instantané $r_t$ permet de caractériser complètement cette courbe. 

Il est donc important que la dynamique de $r_t$ soit à la fois riche pour pouvoir décrire la courbe de rendement observée dans le marché, et suffisemment simple pour que le temps nécessaire pour le calcul ne soit trop long.

Nous pouvons considérer que la courbe de rendement varie dans un espace vectoriel de dimension infinie. Toute tentative de la caracteriser par un nombre fini de paramètre rééls est donc vouée à l'échec.

Le modèle de Hull White a été introduit en 1990.  Un des atouts majeure de ce modèle est la possiblité de simuler la dynamique de $r_t$ par un arbre trinomiale. Ceci étant essentiel pour pricer des produits du type bermuda options
$$ \mathrm{d}r_t =  (\theta(t) - \alpha(t) r_t) \mathrm{d}t + \sigma(t) \mathrm{d} W_t$$
$$ r_t = e^{-\alpha t} r_0 + integral ...$$


\newpage
\section{Le modèle à deux facteurs}
\subsection{Motivation}


Considérons un produit $E$ dont le payoff  dépent  du spread entre un taux d'intêret cumulé entre $0$ et $T_1$ pour le premier et $0$ et $T_2$ pour le second. $E$ dépend donc de la distribution jointe des deux taux.

La figure suivante, tiré du PDF *bla bla bla* reproduit une matrice de corrélation par terme de variations quotidiennes de taux zéro-coupon. Il apparait très clairement que des taux de maturités proches, comme le taux de maturité 3 ans et celui de maturité 4 ans, sont très corrélés, tandis que des taux de maturité éloignées (par exemple le taux 1 mois et le taux 10 ans) le sont très peu:

\IMG{img/tabcorr.png}{Tableau de correlation}{0.3}

Dans la réalité, on observe sur les marchés que les taux à différentes maturités ne sont pas corrélés. Si on regarde le taux 2Y et 10Y

\IMG{img/libor.png}{Libor}{1}

\subsubsection*{Limite des modèle à un seul facteur}
Montrons dans un premier temps pourquoi un modèle à un seul facteur n'est pas suffisant pour pricer ces produits qui dépendent non seulement de la distribution de chaque courbe de taux, mais aussi de leur corrélation. 

La dynamique de $r_t$ dans le modèle de Vasicek est donné par
$$r_t = k(\theta - r_t)  \mathrm{d}t  + \sigma \mathrm{d}W_t$$
La formule analytique du bond zéro coupon est donc
$$P(t, T) = A(t, T) exp(-B(t, T) r_t)$$
En particulier le taux d'intêret cumulé est une donné par une transformation affine du taux instantané:
$$R(t, T) = \frac{ln P(t, T)}{T-t} =: a(t, T) + b(t, T) r_t$$
Le payoff du produit $E$ est donc fonction de la distribution jointe de $R(0, T_1)$ et $R(0, T_2)$. Sauf que:
$$Cor(R(0, T_1), R(0, T_2) = 1$$

On en déduit qu'un choc à $r_t$ agit de la même manière sur toutes les courbes.


Un modèle à un seul facteur ne capture pas ce comportement. Essayons de pallier à ce problème en rajoutons un facteur à ce modèle.

Dans cette section nous considérons un modèle où le taux d'intêret instantanté est donné par une somme de deux facteurs gaussiens centrés et corrélés. Dans ce modèle doit sa popularité au fait que le prix des bond zéron coupon admet une formule exact, ainsi que le prix des caps et des floors.


\begin{align*}
  \rm{d}x &= -\alpha x(t) \rm{d}t + \sigma \rm{d}W^1_t \\
  \rm{d}y &= -\beta y(t) \rm{d}t + \nu \rm{d}W^2_t \\
  \rm{d}r &= x + y 
\end{align*}

\newpage
\section{Approximation de la solution par un arbre binomial}


L'équation (*) s'intègre simplement en:
$$x(t) = x(s) e^{-\alpha (t-s)} +  \sigma \int_s^t e^{- \alpha (t-u)} \rm{d} W^1_u $$
$$y(t) = x(s) e^{-\beta (t-s)} +  \nu \int_s^t e^{- \beta (t-u)} \rm{d} W^2_u $$

\subsubsection{Construction}

Cette méthode a été d'abord suggéré par Hull-White (1994)

On discrétise l’intervalle $[0, T]$ avec les temps $T_i = i \Delta t$, où $\Delta t = \frac{T}{N}$.
Nous donnons une approximation de la dynamique processus $x$ et $y$, par une suite de variable de discretisé $((\widetilde{x}_i, \widetilde{y}_i) \approx (x(i \Delta i), (y(i \Delta t))_i $. Pour celà nous calculerons les deux premier moment de $(x, y)$

$$E(x(t+\Delta t) | F_t) = x(t) e^{-a \Delta t}$$
$$V(x(t+\Delta t) | F_t) = \frac{\sigma^2}{2a} (1 - e^{-2a \Delta t})$$
$$Cov\{x(t+\Delta t), y(t+\Delta t) | F_t \} = \frac{\sigma \nu \rho}{a + b} (1-e^{-(a+b)\Delta t})$$

Pour que le $((\widetilde{x}_i, \widetilde{y}_i)$ et $(x(i \Delta i), (y(i \Delta t))_i $ aient les même moment, la loi de  $((\widetilde{x}_i, \widetilde{y}_i)$  est donnée par:
$$\mathrm{P} \left( \widetilde{x}_{i+1} = \widetilde{x}_i + a \, \mathrm{d}x, \widetilde{y}_{i+1} = \widetilde{y}_i + b \, \mathrm{d}y |  \widetilde{x}_i, \widetilde{y}_i \right) = p^{a, b}( \widetilde{x}_i, \widetilde{y}_i)$$
où 
\begin{itemize}
\item $a, b \in \{-1, +1\}$
\item $p$ est donnée par
\end{itemize}

Nous appelons slice l'ensemble des noeuds qui sont équi distant de la racine. Une slice représente la distribution du processus $( \widetilde{x}_i, \widetilde{y}_i)$ à un instant donné.

Le pricing se fait en deux temps:
\begin{itemize}
\item On diffuse le processus $(\widetilde{x}, \widetilde{y})$ dans l'arbre en prenant soin de calculer la probabilité de transition d'un état à un autre
\item On ``drawback'' dans l'arbre en partant de la date à laquelle on fait le payoff, en ***discountant***
\end{itemize}

  \subsubsection*{Petite discussion sur la courbe d'actualisation vs la courbe de diffusion}
  Avant la crise de 2008, il était d'usage courant que les banques considèrent le taux Libor comme reflétant la réalité du marché de crédit inter-bancaire. Le taux est publié quotidiennement par
taux réél auquel les banques sont prête à se prétter de l'argent est appelé taux OIS (Overnight Index Swap).  
Pendant la crise, le spread entre LIBOR et OIS était si grand qu'il devenait impossible à ignorer. Depuis tous les modèles de taux intégrent deux courbe, une pour la diffusion (LIBOR par exemple) et une autre pour l'actualisation (OIS).
Dans le développement de cet article, nous ignorons cette différence.
  
\subsection{Améliorations}
Si nous implémentons l'abre de façon naîve, le nombre de noeuds augmente de façon exponentielle en fonction du nombre de pas de temps. En pratique ceci est problématique et conduit vite à une saturation de mémoire. Dans l'exemple simplifié ci-dessus nous traçons l'arbre de diffusion du premier facteur ($x(t)$). A chaque pas de temps le nombre de noeuds double, ie pour $n$ pas de temps, nous nous retrouvons avec $2^n$ noeuds pour un facteur, ou $4^n$ pour deux. 

Remarquer que $(x, y)_i$tilde est un processus markovien homogène à valeurs discrètes nous permet d'optimiser la simulation de l'arbre. En effet, il nous suffit de calculer la table de transition une fois au début du programme et de la réutiliser pour avancer/reculer dans le temps. 

%%% Local Variables:
%%% mode: latex
%%% TeX-master: t
%%% End:

% Define styles for bags and leafs
\tikzstyle{bag} = [text width=2em, text centered]
\tikzstyle{end} = []
\begin{figure}[H]
  \centering
\begin{tikzpicture}[sloped]
  \node (a) at ( 0,0) [bag] {$0$};
  \node (b) at ( 4,-1.5) [bag] {$- \sigma \Delta t$};
  \node (c) at ( 4,1.5) [bag] {$+ \sigma \Delta t$};
  \node (d) at ( 8,-3) [bag] {$-2 \sigma \Delta t$};
  \node (e1) at ( 8,0.5) [bag] {$+ 0 \sigma \Delta t$};
  \node (e2) at ( 8,-0.5) [bag] {$+ 0 \sigma \Delta t$};
  \node (f) at ( 8,3) [bag] {$+ 2 \sigma \Delta t$};
  
  \draw [->] (a) to node [below] {$p^-$} (b);
  \draw [->] (a) to node [above] {$p^+$} (c);
  \draw [->] (c) to node [below] {$p^+$} (f);
  \draw [->] (c) to node [above] {$p^-$} (e1);
  \draw [->] (b) to node [below] {$p^+$} (e2);
  \draw [->] (b) to node [above] {$p^-$} (d);
\end{tikzpicture}
\caption{Arbre construit de façon naïve}
\label{oldtree}
\end{figure}




\IMG{img/slice.png}{Slice}{0.5}
\IMG{img/pending.jpg}{Cache grind avant}{0.2}
\IMG{img/pending.jpg}{Cache grind avant}{0.2}

Une autre améliroation possible est de trouver une formule analytique pour certain produits. En effet, si nous reprenons l'exemple d'un cancellable spread option 2Y10Y dont la maturité est dans 5 ans, nous devrions normalement construite l'arbre jusqu'en 2030 pour avoir le taux 10Y en 2020. Nous pouvons éviter celà en fournissant directement une formule exacte pour les zéro coupons.

\subsubsection{Formule exacte}
Cette partie est fortement insipiré de *BLA BLA BLA*

On rappelle l'expression du prix du bond zéron coupon sous la mesure risque neutre $Q$
$$P(t, T) = \espr{ e^{-\int_t^T r_u \rm{d}u}} $$

Notons $I(t, T) := \int_t^T x(u) + y(u) \rm{d}u$, et montrons que conditionnellement à l'information accumulé jusqu'au temps $t$, c'est une variable normale d'esperance $M(t, T)$ et de variance $V(t, T)$ où:
$$M(t, T) := ...$$
$$V(t, T) := ...$$

Nous avons donc
$$P(t, T) = exp  \left( \espr{ -\int_t^T r_u \rm{d}u} + \frac{1}{2} Var \left( {-\int_t^T r_u \rm{d}u} \right)  \right) $$

Nous utiliserons cette formule directement dans le pricer, ce qui nous évitera de construire l'arbre jusqu'à la date de maturité du dernier zéro coupon.

\subsubsection{Ajout d'un shift déterministe}
Le modèle, tel que développé jusqu'a présent, présente un inconvénient majeur: à tout instant, $r_t$ est symétriquement distribué autour de $0$. Ceci ne correspond pas à la réalité, puisque les taux négatifs ne sont observé dans les marchés que dans de très rares circonstance ( en Europe après crise inter-bancaire de 2009, Au Japon après des années de déflation ). Une autre raison est que le modèle ne permet pas de retrouver les prix des bonds zéro coupon.

Pour pallier à ce problème, nous rajoutons une fonction $\phi$ au taux $r_t$. La fonction déterministe $\phi(t)$ permet de fitter exactement la courbe de taux observée. Dans la partie ``Calibraion nous'' nous verrons comment calculer cette fonction à partir des prix de bonds zéro coupon.

En prenant en compte ce changement, le prix du bond zéro coupon devient:

$$P(t, T) = \espr{ e^{-r_t + \phi(t)}}$$

Nous prendrons soin de modifier l'étape de ``draw back'' dans l'abre en changeant le facteur d'actualisation.

Le processus est markovien

\subsubsection{Optimiser la taille des slices }
La taille de la slice augmente linéairement avec le temps. Ce n'est pas raisonnable.
Comment connecter les slices entre elles


\subsubsection{ Paramètres dépendant du temps}

Dans la partie précédente, les paramètres $\sigma \nu \alpha \beta$ étaient constantes. Considérer des variables qui dépendent  du temps permet au modèle plus de flexibilité pour fitter les données de marché. Voir la partie calibration.

Deux problèmes cependant:

\begin{itemize}
\item Le calcul est beaucoup plus long,
\item possibilité de sur fitter les données historique du marché, ce qui affecte négativement le pouvoir prédicitive du modèle
\end{itemize}

<Calcul>

On vérifie expérimentalement que le gain est significatif

\IMG{img/pending.jpg}{Cache grind avant}{0.2}
\IMG{img/pending.jpg}{Cache grind apres}{0.2}

\section{Résultats}

Une fois toutes ces modifications prises en compte, nous pouvons *BLA BLA BLA*
Paragraphe sur la taille des slice, ellipsoid
\IMG{img/pending.jpg}{Slice 2D}{0.2}


\begin{itemize}
\item Discrétisation
\item Construction de l'abre
\end{itemize}


\subsection{Performance}
L'arbre est long mais beaucoup plus puissant
Imperfections de l'arbre:
\begin{itemize}
\item  bornee
\item  discretisation
\item  probabilite negative
\end{itemize}
Monte Carlo
Limitation
Closed Form

Le temps d'execution:
Arbre
Closed form

\newpage

\section{Application: calibration et pricing}
Notre modèle possède à un certains nombres de paramètres libre que nous devons fixer. Pour celà, nous choisissons un certain nombre d'actifs tradables dans le marché, dont le prix est donc connus, que nous appelerons benchmark. Nous essayerons ensuite de trouver les paramètres qui reproduisent le mieux ces prix là. Cette procédure est appelé calibration.
Une question naturelle qui se pose est de savoir quels actifs choisir pour la calibration. Il existe plusieurs réponses possibles, en pratique on essaye de trouver un produit à la fois simple et liquide.

Dans notre cas il est indispensable que le modèle puissent retrouver les prix des bond zéro coupons.
Idéalement notre benchmarks est une ensemble de caplets. Les caplets ne sont pas tradés en tant que tel sur le marché, nous n'avons accès qu'à des caps. => Stripping

Le modèle à 2F permet de caputrer le hump de la courbe de rendement
Le nombre de paramètre est fini (5) => pas de overfitting


\begin{itemize}
\item $h(t)$ pour reconstruire la yield curve $\sigma \rho \nu$ pour
\item  matcher la surface
\end{itemize}

\subsection{Calibration du drift}
Le modèle gaussien à deux facteurs est calibré sur la
courbe P M (0, T ), T > 0 de prix d’obligations zéro-coupon observés sur le
marché si et seulement si $\phi$ est définie par :
$$\Phi = ...$$

Cependant, l’arbre ainsi simulé ne redonnera par exactement les prix des
obligations zéro-coupon P (0, T i ). En effet, dans un arbre le taux simulé est
considéré constant sur la période $[T_i, T_{i+1}$, donc tout se passe comme
si l’arbre simulait en fait un taux zéro-coupon R. En d’autres termes, le prix
d’une obligation zéro-coupon de maturité $T_1$ vaut
$P(0, T_1 ) = e^{-R(0, T_1) \Delta t}$
mais ce prix calculé directement sur l’arbre s’écrira $e^{-r_0\Delta t}$ .
On propose donc ici de calibrer récursivement les $\phi_i$ afin de
retrouver les prix des obligations zéro-coupon directement sur l’arbre. 
$\Pi_j$ state price (arrow debrew) (paye 1 si le noeud $(t_n, j)$ est atteint.

$$D_j(h_n) := \frac{1}{1 + r(t_n)(t_{n+1} - t_n)} $$ Le discount factor
$$ \Pi_j(t_{n+1}) = \sum_j \Pi_j(t_n) p_{j, j'}(t_n) D_{j'}(h_n)$$
$$ \sum \Pi_j D_j(h_n) = P(0, t_{n+1}) $$

Developement de taylor => trouver $h_n$

\subsection{Méthode de calibration de la surface - Méthode d'optimisation}

Les traders préfère aiment bien avoir un controle fin sur le modèle qui reflète leur sensation sur le marché. Pour celà, le modèle doit êtres paramètrble, les paramètres doivent avoir un sens/être compris par les traders.

Au lieu d'avoir un paramète par maturité, G2++ il y a 5 paramètres


\begin{itemize}
\item On s'autorise un intervalle pour les paramètres
\item On utilise une grille (définie par le pas) pour définir les valeurs
  autorisées pour chaque paramètre. (Tradeoff entre pas petite grande
  précision et temps de calculs)
\item On calcule le prix des caplets associés
\item On choisit les paramètre qui reflètent le mieux les prix du
  marché.
\end{itemize}

Maintenant que nous connaissons la dynamique de $P(t, T)$, nous pouvons fournir une formule explicite pour le prix des caplets.

En effet, sous la mesure forward neutre $Q_T$, l'equation vérifiér  par $(x, y)$ est:

\begin{align*}
  \rm{d}x &= -\alpha x(t) \rm{d}t - Drift_x + \sigma \rm{d}W^1_t \\
  \rm{d}y &= -\beta y(t) \rm{d}t - Drift_y + \nu \rm{d}W^2_t \\
\end{align*}

qui a donc pour solution:
\begin{align*}
  x(t) &= -\alpha x(t) \rm{d}t - Drift_x + \sigma \rm{d}W^1_t \\
  y(t) &= -\beta y(t) \rm{d}t - Drift_y + \nu \rm{d}W^2_t \\
\end{align*}

Nous rappeleons l'expression du caplet sour la mesure $Q_T$
$$ZBP = \Qespr{Q_T}{ ... } $$

$P(t, T)$ admet une distribution normale sous $Q_T$ conditionnellement à $F_t$, 
Le prix théorique:

$$ZBC = P(t, T) N( d_1 ) - P(t, T) K N(d_2)$$
$$d_{1/2} = \frac{ln \frac{P(t, S)}{KP(t, T)}}{V_p} +- \frac{1}{2} V_P $$

On a 5 paramètres à optimiser
On calibre les caplets

Les caplets ne sont pas directement disponilbe sur les marche
On calibre les swaptions/caps

\begin{itemize}
\item Le probleme de calbration est un probleme d'optimisation Courbe
  de vol implicite :
\item minimisation de l'erreur L2
\end{itemize}
\IMG{img/capsurf.png}{Cap surface}{0.5}

\subsubsection{Un mot sur le multi threading - gpu?}
Le modèle permet de 
multi process

%%%% Local Variables:
%%% mode: latex
%%% TeX-master: "main"
%%% End:

\appendix

{\newgeometry{left=0.8in,right=0.8in,top=1in,bottom=1in}
\chapter{Code de simulation en Python}
\inputminted{python}{code/tree.py}
\chapter{Formule exacte des obligations zéro coupon pour des paramètres dépendant du temps}

Dans cette partie nous détaillerons le calcul de prix d'obligation zéro coupon dans le modèle de Hull-White à deux facteurs dans le cas où les paramètre sont dépendents du temps, ou plus précisément, constants par morceaux.

Le modèle est toujours markovien, c'est à dire que nous pouvons toujours écrire $P(t, T)$ comme une fonction détérministe de $(x(t), y(t)$:

$$P(t, T) := e^{\int_t^T \phi_s \rm{d}s - M_x(t, T) x(t) - M_y(t, T) y(t) + \frac{1}{2} V(t, T)}$$

Nous calculerons ici $M_x$, $M_y$ et $V$  

\subsection*{Rappel du modèle}
$$\rm{d}x_t = -\beta_x x_t \rm{d}t +  \sigma_x \rm{d} W^1_t $$
$$\rm{d}y_t = -\beta_y y_t \rm{d}t +  \sigma_y \rm{d} W^2_t $$
$$r_t =  \phi(t) + x_t  + y_t $$
$$\rho_t = <\rm{d} W^1_t, \rm{d} W^2_t>$$

\subsection*{Cas particulier: modèle à un seul facteur}
Nous commencerons par le cas particulier où $y(t) = 0$, l'équation différentielle vérifiée par $x(t)$ s'intègre facilement en:
$$x(t) = \sum_{t_i < t} \sigma_i  \int_{t_i}^{t \wedge t_{i+1}}  e^{- a (t \wedge t_{i+1}-s)}   \rm{d} W_1(s) $$

Nous devons maintenant intégrer la fonction $x$ entre $t_0$ et $t_f$ en la décomposant en somme d'intégrales entre les instant $t_i$ et $t_{i+1}$ où tous les paramètres sont constants et l'intégrale se calcule facilement.

\begin{align*}
\int_{t_0}^{t_f} x(t) dt &= \sum_i \int_{t_i}^{t_{i+1}} x(t) dt \\
&= \sum_i \frac{1 - e^{-\beta_i (t_{i+1} - t_i) }}{ \beta_i} x(t_i)
+ \frac{\sigma_i}{\beta_i} \int_{t_i}^{t_{i+1}} (1 - e^{-\beta_i (t_{i+1} - u)}) dW_u \\
&= \sum_i \frac{1 - e^{-\beta_i (t_{i+1} - t_i) }}{ \beta_i} e^{-\int_{t_0}^{t_i} \beta} x(t_0) \\
&+  \sum_i \frac{1 - e^{-\beta_i (t_{i+1} - t_i) }}{ \beta_i} \int_{t_0}^{t_i} \sigma_u e^{-\int_u^{t_i} \beta} dW_u
+ \frac{\sigma_i}{\beta_i} \int_{t_i}^{t_{i+1}} (1 - e^{-\beta_i (t_{i+1} - u)}) dW_u \\
&=: M(t_0, t_f) x(t_0) + v(t_0, t_f) \sim \mathcal{N}( M(t_0, t_f), V(t_0, t_f))
\end{align*}
où nous avons noté: $V(t_0, t_f) := Var( \int_{t_0}^{t_f} x ) = Var( v(t_0, t_f))$

d'où
\begin{align*}
P(t, T) &= E\left[ exp \left\{  -\int_t^T h(u) + E(-\int_t^T x) - \frac{1}{2} Var(-\int_t^T x) \!  \right\} \right]
\end{align*}

Simplifions l'écriture de $V(t_0, t_f)$
\begin{align*}
v(t_0,t_f) &:=
\sum_{i, t_0 \leq t_i \leq t_{i+1} \leq t_f }
\frac{1 - e^{-\beta_i (t_{i+1} - t_i) }}{ \beta_i} \int_{t_0}^{t_i} \sigma_u e^{-\int_u^{t_i} \beta} dW_u
+ \frac{\sigma_i}{\beta_i} \int_{t_i}^{t_{i+1}} (1 - e^{-\beta_i (t_{i+1} - u)}) dW_u \\
&=
\sum_{i}
\frac{1-e^{- \int_{t_i}^{t_{i+1}} \beta}}{\beta_i}
\int_{t}^{t_{i}} \sigma_u e^{-\int_u^{t_i} \beta} dW_u
+
\frac{\sigma_i}{\beta_i} \int_{t_i}^{t_{i+1}} 1-e^{-\int_u^{t_{i+1}} \beta} dWu  \\
&=
\int_{t_0}^{t_f} \frac{\sigma_u}{\beta_u} dW_u
+
\sum_{i }
\int_{t}^{t_{i}} \frac{\sigma_u}{\beta_i} e^{-\int_u^{t_i} \beta} dW_u
- \int_{t}^{t_{i}} \frac{\sigma_u}{\beta_i} e^{-\int_u^{t_{i+1}} \beta} dW_u
- \frac{\sigma_i}{\beta_i} \int_{t_i}^{t_{i+1}} e^{-\int_u^{t_{i+1}} \beta} du
\\
&=
\int_{t_0}^{t_f} \frac{\sigma_u}{\beta_u} dW_u
+
\sum_{i }
 \frac{ \int_{t}^{t_{i}}\sigma_u e^{-\int_u^{t_i} \beta} dW_u }{\beta_i}
- \frac{ \int_{t}^{t_{i+1}}\sigma_u e^{-\int_u^{t_{i+1}} \beta} dW_u}{\beta_i}
\\
&=
\int_{t_0}^{t_f} \frac{\sigma_u}{\beta_u} dW_u
+
\sum_{i}
\frac{K_i - K_{i+1}}{\beta_i}
& \text{with $K_i = \int_{t_0}^{t_{i}}\sigma_u e^{-\int_u^{t_i} \beta} dW_u $}
\\
&=
\int_{t_0}^{t_f} \frac{\sigma_u}{\beta_u} dW_u
+
\sum_{i}
(\frac{1}{\beta_i} - \frac{1}{\beta_{i+1}}) K_i\\
&=
\int_{t_0}^{t_f} \frac{\sigma_u}{\beta_u} dW_u
+
\sum_{i=1..n}c_i K_i
&\text{avec  $c_i = \frac{1}{\beta_i} - \frac{1}{\beta_{i-1}}$ and $\beta_{n} = \infty$}
\end{align*}

Nous sommes intéressés par la variance de cette quantité là:

\begin{align*}
Var(t_0, t_f) &:= var( v(t_0,t_f)| F_{t_0}  ) \\
&= <\int_{t_0}^{t_f} \frac{\sigma_u}{\beta_u} dW_u, \int_{t_0}^{t_f} \frac{\sigma_u}{\beta_u} dW_u>
+ 2 \sum_i c_i  < K_i , \int_{t_0}^{t_f} \frac{\sigma_u}{\beta_u} dW_u>
+  \sum_{i, j} c_i c_j <  K_i,   K_j> \\
&= \omega
+ 2 \sum_i c_i  \alpha_i
+  \sum_{i, j}c_i c_j \gamma_{ij} \\
&= \omega + 2 \sum_i c_i  \alpha_i
+  2 \sum_{i < j} c_i c_j e^{-\int_{t_i}^{t_j} \beta} \gamma_{i}
+ \sum_{i } c_i^2  \gamma_{i}
\\
&= \omega + 2 \sum_i c_i  \alpha_i
+ 2 \sum_{i=1..n} \gamma_{i} \frac{c_i}{I_i} \left( \sum_{j = i+1...n} I_j c_j \right)
+ \sum_{i } c_i^2  \gamma_{i}\\
&= \omega + 2 \sum_i c_i  \alpha_i
+ 2 \sum_{i=1..n} \gamma_{i} \frac{c_i}{I_i} \left( S_n - S_i \right)
+ \sum_{i } c_i^2  \gamma_{i}
\end{align*}

Où
\begin{align*}
  I_i &:= e^{-\int^{t_i}_0 \beta} \\
  S_i &:= \sum_{j = 0...i} I_j c_j \\
  \omega &:= \sum_i  (\frac{\sigma_i}{\beta_i})^2 (t_{i+1} - t_i)
\end{align*}

Et les suite $\alpha_i$ et $\gamma_i$ sont définies par réccurence:
  \begin{align*}
    \alpha_{i+1} &:= e^{-\beta_i (t_{i+1} - t_i)} \alpha_i + (\frac{\sigma_i}{\beta_i})^2 (1 - e^{-\beta_i(t_{i+1} - t_i)}) \\
    \gamma_{i+1} &:= e^{-2 \beta_i (t_{i+1} - t_i)} \gamma_i + \frac{\sigma_i^2}{2 \beta_i} (1 - e^{-2 \beta_i(t_{i+1} - t_i)})\\
\end{align*}
    
Dans la section suivante on détaille le calcul de $\alpha$, $\gamma$ et $\omega$

Cette formule permet de calculer $V$ en temps linéaire (ie $O(t_f - t_0)$)
\subsection*{Calculations}

Pour $i < j$
\begin{align*}
\gamma_{i, j} &:= <K_i, K_j>  \\
&= < \int_{t}^{t_i}\sigma_u e^{-\int_u^{t_i} \beta} dW_u,
\int_{t}^{t_j}\sigma_u e^{-\int_u^{t_j} \beta} dW_u > \\
&= e^{-\int_{t_i}^{t_j} \beta} \int_t^{t_i} (\sigma_u e^{-\int_u^{t_i} \beta})^2 du \\
&= e^{-\int_{t_i}^{t_j} \beta} \int_t^{t_i} \sigma_u^2 e^{-2 \int_u^{t_i} \beta} du \\
&= e^{-\int_{t_i}^{t_j} \beta} \gamma_{i, i}
\end{align*}

\begin{align*}
\gamma_{i+1} &:= \gamma_{i+1, i+1} \\
&= \int_t^{t_{i+1}} \sigma_u^2 e^{-2 \int_u^{t_{i+1}} \beta} du \\
&= e^{-2 \beta_i (t_{i+1} - t_i)} \gamma_i + \int_{t_i}^{t_{i+1}} \sigma_i^2 e^{-2 \beta_i(t_{i+1} - u)} du \\
&= e^{-2 \beta_i (t_{i+1} - t_i)} \gamma_i + \frac{\sigma_i^2}{2 \beta_i} (1 - e^{-2 \beta_i(t_{i+1} - t_i)})
\end{align*}

\begin{align*}
\alpha_i &:=
<K_i, \int_{t_0}^{t_f} \frac{\sigma_u}{\beta_u} dW_u> \\
&=
< \int_{t}^{t_i}\sigma_u e^{-\int_u^{t_i} \beta} dW_u,
\int_{t_0}^{t_f} \frac{\sigma_u}{\beta_u} dW_u > \\
&=   \int_t^{t_i} \frac{\sigma_u^2}{\beta_u} e^{-\int_u^{t_i} \beta} du \\
\end{align*}

\begin{align*}
\alpha_{i+1}
&= e^{-\beta_i (t_{i+1} - t_i)} \alpha_i + \int_{t_i}^{t_{i+1}} \frac{\sigma_u^2}{\beta_u} e^{-\beta_i(t_{i+1} - u)} du\\
&= e^{-\beta_i (t_{i+1} - t_i)} \alpha_i + (\frac{\sigma_i}{\beta_i})^2 (1 - e^{-\beta_i(t_{i+1} - t_i)})
\end{align*}


\begin{align*}
\omega &:=
<\int_{t_0}^{t_f} \frac{\sigma_u}{\beta_u} dW_u, \int_{t_0}^{t_f} \frac{\sigma_u}{\beta_u} dW_u>\\
&= \int_{t_0}^{t_f} (\frac{\sigma_u}{\beta_u})^2 du \\
&= \sum_i  (\frac{\sigma_i}{\beta_i})^2 (t_{i+1} - t_i)
\end{align*}

\subsection*{Le modèle à deux facteurs}
Nous revenons au modèle original à deux facteurs. Les paramètre relatifs au facteur $x$ (resp. $y$) seront dénoté par un x (resp. y) en exposant.

L'espérance étant linéaire, et la variance quadratique, 
$M(t_0, t_f)$ est remplacée par $M^x x + M^y y$, et $V(t_0, t_f)$ par $V^{xx} + V^{yy} + 2 V^{xy}$, de sorte que:

$$P(t_0, T_f) = exp(-\int_{t_0}^{t_f} \Phi - M^x(t_0, t_f) x(t_0) - M^y(t_0, t_f) y(t_0) + \frac{V(t_0, t_f)}{2})$$

\begin{align*} V^{xy} &:= <
\int_{t_0}^{t_f} \frac{\sigma^x_u}{\beta_u^x} dW_u^x+\sum_{i=1..n}c_i^x K_i^x,
\int_{t_0}^{t_f} \frac{\sigma^y_u}{\beta_u^y} dW_u^y+\sum_{i=1..n}c_i^y K_i^y> \\
&= \int_{t_0}^{t_f} \frac{\sigma_u^x \sigma_u^y}{\beta_u^x \beta_u^y} \rho_u du
+ \sum_{ij} c_i^x c_j^y <K_i^x, K_i^y>
+ \sum_i c_i^x <K_i^x \int_{t_0}^{t_f} \frac{\sigma^y_u}{\beta_u^y} dW_u^y> + c_i^y <K_i^y \int_{t_0}^{t_f} \frac{\sigma^x_u}{\beta_u^x} dW_u^x>\\
&= \omega^{x,y} + \sum_{ij} c_i^x c_j^y \gamma_{ij}^{xy}
+ \sum_i c_i^x \alpha_i^x + c_i^y \alpha_i^y
\end{align*}

avec comme pour le cas à un seul facteur:
$$\alpha^x_{i+1} = e^{-\beta^x_i(t_{i+1} - t_i)} \alpha^x_i + \rho_i \frac{\sigma^x_i \sigma^y_i}{\beta^x_i \beta^y_i} (1 - e^{-\beta^x_i(t_{i+1} - t_i)})$$
$$\gamma_{i+1} = e^{- (\beta^x_i+\beta^y_i) (t_{i+1} - t_i)} \gamma_i + \rho_i \frac{\sigma_i^x \sigma_i^y}{\beta_i^x + \beta_i^y} (1 - e^{- (\beta^x_i+\beta^y_i)(t_{i+1} - t_i)})$$
$$\omega := \sum_i \rho_i  \frac{\sigma_i^x \sigma_i^y}{\beta_i^x \beta_i^y} (t_{i+1} - t_i)$$

\iffalse
With $i \leq j$:
\begin{align*}
 \gamma_{ij}^{xy} &:= <K_i^x, K_i^y> \\
  &= <\int_t^{t_i} \sigma_u^x e^{-\int_t^{t_i} \beta^x} dW^x, \int_t^{t_j} \sigma_u^y e^{-\int_t^{t_j} \beta^y} dW^y> \\
  &= e^{-\int_{t_i}^{t_j} \beta_y} \int_t^{t_i} \sigma_u^x \sigma_u^y e^{-\int_t^{t_i} \beta^x + \beta_y} \rho_u du \\
&= e^{-\int_{t_i}^{t_j} \beta_y} \gamma_{ii}
\end{align*}
\fi






\end{document}


%%% Local Variables:
%%% mode: latex
%%% TeX-master: t
%%% End:
