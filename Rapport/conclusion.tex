\chapter*{Conclusion}
Nous avons étudié le modèle de Hull White. Nous avons vu qu'un seul facteur ne suffit pas pour expliquer la dépendance entre les taux à différentes maturités. Nous nous sommes intéressés au pricing de produits exotiques en utilisant un arbre binomial. Ce procédé étant lent en calcul, nous avons utilisé une formule exacte pour tous les produits qui en admettent une.

Nous étions intéressés par la méthode de calibration, en se basant sur les caps et les swaptions. Ces derniers dépendent de la structure de corrélation entre différents taux, ce qui nous a permis de confirmer notre intuition sur la validité du modèle à deux facteurs.

Finalement ce stage m'a donné la possibilité d'étudier directement l'aspect de modélisation et d'implémentation d'un modèle financier, ainsi que le temps d'étudier la partie théorique qui lui est associée. Tout cela permet de voir quels sont les avantages, ses faiblesse,  et les innovations qu'il présente.

%%% Local Variables:
%%% mode: latex
%%% TeX-master: "main"
%%% End:
