\chapter{Conclusion}

Un partie très importante dans la valorisation des produits financiers est dédidée à L’étude et l’implémentation des modèles de taux.
Nous étions intéressés par la méthode de calibration et de modélisation du modèle Gaussien G2++ à deux facteurs. A cause des limites des modèles à un seul facteur, nous avons estimé que le modèle à deux facteurs peut améliorer la capacité de modélisation des modèles de taux.
Les résultats aposteriori confirme notre intuition.


%%% Local Variables:
%%% mode: latex
%%% TeX-master: "main"
%%% End:
