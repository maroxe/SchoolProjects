\chapter{Formule exacte des obligations zéro coupon pour des paramètres dépendant du temps}

Dans cette partie nous détaillerons le calcul de prix d'obligation zéro coupon dans le modèle de Hull-White à deux facteurs dans le cas où les paramètre sont dépendents du temps, ou plus précisément, constants par morceaux.

Le modèle est toujours markovien, c'est à dire que nous pouvons toujours écrire $P(t, T)$ comme une fonction détérministe de $(x(t), y(t)$:

$$P(t, T) := e^{\int_t^T \phi_s \rm{d}s - M_x(t, T) x(t) - M_y(t, T) y(t) + \frac{1}{2} V(t, T)}$$

Nous calculerons ici $M_x$, $M_y$ et $V$  

\subsection*{Rappel du modèle}
$$\rm{d}x_t = -\beta_x x_t \rm{d}t +  \sigma_x \rm{d} W^1_t $$
$$\rm{d}y_t = -\beta_y y_t \rm{d}t +  \sigma_y \rm{d} W^2_t $$
$$r_t =  \phi(t) + x_t  + y_t $$
$$\rho_t = <\rm{d} W^1_t, \rm{d} W^2_t>$$

\subsection*{Cas particulier: modèle à un seul facteur}
Nous commencerons par le cas particulier où $y(t) = 0$, l'équation différentielle vérifiée par $x(t)$ s'intègre facilement en:
$$x(t) = \sum_{t_i < t} \sigma_i  \int_{t_i}^{t \wedge t_{i+1}}  e^{- a (t \wedge t_{i+1}-s)}   \rm{d} W_1(s) $$

Nous devons maintenant intégrer la fonction $x$ entre $t_0$ et $t_f$ en la décomposant en somme d'intégrales entre les instant $t_i$ et $t_{i+1}$ où tous les paramètres sont constants et l'intégrale se calcule facilement.

\begin{align*}
\int_{t_0}^{t_f} x(t) dt &= \sum_i \int_{t_i}^{t_{i+1}} x(t) dt \\
&= \sum_i \frac{1 - e^{-\beta_i (t_{i+1} - t_i) }}{ \beta_i} x(t_i)
+ \frac{\sigma_i}{\beta_i} \int_{t_i}^{t_{i+1}} (1 - e^{-\beta_i (t_{i+1} - u)}) dW_u \\
&= \sum_i \frac{1 - e^{-\beta_i (t_{i+1} - t_i) }}{ \beta_i} e^{-\int_{t_0}^{t_i} \beta} x(t_0) \\
&+  \sum_i \frac{1 - e^{-\beta_i (t_{i+1} - t_i) }}{ \beta_i} \int_{t_0}^{t_i} \sigma_u e^{-\int_u^{t_i} \beta} dW_u
+ \frac{\sigma_i}{\beta_i} \int_{t_i}^{t_{i+1}} (1 - e^{-\beta_i (t_{i+1} - u)}) dW_u \\
&=: M(t_0, t_f) x(t_0) + v(t_0, t_f) \sim \mathcal{N}( M(t_0, t_f), V(t_0, t_f))
\end{align*}
où nous avons noté: $V(t_0, t_f) := Var( \int_{t_0}^{t_f} x ) = Var( v(t_0, t_f))$

d'où
\begin{align*}
P(t, T) &= E\left[ exp \left\{  -\int_t^T h(u) + E(-\int_t^T x) - \frac{1}{2} Var(-\int_t^T x) \!  \right\} \right]
\end{align*}

Simplifions l'écriture de $V(t_0, t_f)$
\begin{align*}
v(t_0,t_f) &:=
\sum_{i, t_0 \leq t_i \leq t_{i+1} \leq t_f }
\frac{1 - e^{-\beta_i (t_{i+1} - t_i) }}{ \beta_i} \int_{t_0}^{t_i} \sigma_u e^{-\int_u^{t_i} \beta} dW_u
+ \frac{\sigma_i}{\beta_i} \int_{t_i}^{t_{i+1}} (1 - e^{-\beta_i (t_{i+1} - u)}) dW_u \\
&=
\sum_{i}
\frac{1-e^{- \int_{t_i}^{t_{i+1}} \beta}}{\beta_i}
\int_{t}^{t_{i}} \sigma_u e^{-\int_u^{t_i} \beta} dW_u
+
\frac{\sigma_i}{\beta_i} \int_{t_i}^{t_{i+1}} 1-e^{-\int_u^{t_{i+1}} \beta} dWu  \\
&=
\int_{t_0}^{t_f} \frac{\sigma_u}{\beta_u} dW_u
+
\sum_{i }
\int_{t}^{t_{i}} \frac{\sigma_u}{\beta_i} e^{-\int_u^{t_i} \beta} dW_u
- \int_{t}^{t_{i}} \frac{\sigma_u}{\beta_i} e^{-\int_u^{t_{i+1}} \beta} dW_u
- \frac{\sigma_i}{\beta_i} \int_{t_i}^{t_{i+1}} e^{-\int_u^{t_{i+1}} \beta} du
\\
&=
\int_{t_0}^{t_f} \frac{\sigma_u}{\beta_u} dW_u
+
\sum_{i }
 \frac{ \int_{t}^{t_{i}}\sigma_u e^{-\int_u^{t_i} \beta} dW_u }{\beta_i}
- \frac{ \int_{t}^{t_{i+1}}\sigma_u e^{-\int_u^{t_{i+1}} \beta} dW_u}{\beta_i}
\\
&=
\int_{t_0}^{t_f} \frac{\sigma_u}{\beta_u} dW_u
+
\sum_{i}
\frac{K_i - K_{i+1}}{\beta_i}
& \text{with $K_i = \int_{t_0}^{t_{i}}\sigma_u e^{-\int_u^{t_i} \beta} dW_u $}
\\
&=
\int_{t_0}^{t_f} \frac{\sigma_u}{\beta_u} dW_u
+
\sum_{i}
(\frac{1}{\beta_i} - \frac{1}{\beta_{i+1}}) K_i\\
&=
\int_{t_0}^{t_f} \frac{\sigma_u}{\beta_u} dW_u
+
\sum_{i=1..n}c_i K_i
&\text{avec  $c_i = \frac{1}{\beta_i} - \frac{1}{\beta_{i-1}}$ and $\beta_{n} = \infty$}
\end{align*}

Nous sommes intéressés par la variance de cette quantité là:

\begin{align*}
Var(t_0, t_f) &:= var( v(t_0,t_f)| F_{t_0}  ) \\
&= <\int_{t_0}^{t_f} \frac{\sigma_u}{\beta_u} dW_u, \int_{t_0}^{t_f} \frac{\sigma_u}{\beta_u} dW_u>
+ 2 \sum_i c_i  < K_i , \int_{t_0}^{t_f} \frac{\sigma_u}{\beta_u} dW_u>
+  \sum_{i, j} c_i c_j <  K_i,   K_j> \\
&= \omega
+ 2 \sum_i c_i  \alpha_i
+  \sum_{i, j}c_i c_j \gamma_{ij} \\
&= \omega + 2 \sum_i c_i  \alpha_i
+  2 \sum_{i < j} c_i c_j e^{-\int_{t_i}^{t_j} \beta} \gamma_{i}
+ \sum_{i } c_i^2  \gamma_{i}
\\
&= \omega + 2 \sum_i c_i  \alpha_i
+ 2 \sum_{i=1..n} \gamma_{i} \frac{c_i}{I_i} \left( \sum_{j = i+1...n} I_j c_j \right)
+ \sum_{i } c_i^2  \gamma_{i}\\
&= \omega + 2 \sum_i c_i  \alpha_i
+ 2 \sum_{i=1..n} \gamma_{i} \frac{c_i}{I_i} \left( S_n - S_i \right)
+ \sum_{i } c_i^2  \gamma_{i}
\end{align*}

Où
\begin{align*}
  I_i &:= e^{-\int^{t_i}_0 \beta} \\
  S_i &:= \sum_{j = 0...i} I_j c_j \\
  \omega &:= \sum_i  (\frac{\sigma_i}{\beta_i})^2 (t_{i+1} - t_i)
\end{align*}

Et les suite $\alpha_i$ et $\gamma_i$ sont définies par réccurence:
  \begin{align*}
    \alpha_{i+1} &:= e^{-\beta_i (t_{i+1} - t_i)} \alpha_i + (\frac{\sigma_i}{\beta_i})^2 (1 - e^{-\beta_i(t_{i+1} - t_i)}) \\
    \gamma_{i+1} &:= e^{-2 \beta_i (t_{i+1} - t_i)} \gamma_i + \frac{\sigma_i^2}{2 \beta_i} (1 - e^{-2 \beta_i(t_{i+1} - t_i)})\\
\end{align*}
    
Dans la section suivante on détaille le calcul de $\alpha$, $\gamma$ et $\omega$

Cette formule permet de calculer $V$ en temps linéaire (ie $O(t_f - t_0)$)
\subsection*{Calculations}

Pour $i < j$
\begin{align*}
\gamma_{i, j} &:= <K_i, K_j>  \\
&= < \int_{t}^{t_i}\sigma_u e^{-\int_u^{t_i} \beta} dW_u,
\int_{t}^{t_j}\sigma_u e^{-\int_u^{t_j} \beta} dW_u > \\
&= e^{-\int_{t_i}^{t_j} \beta} \int_t^{t_i} (\sigma_u e^{-\int_u^{t_i} \beta})^2 du \\
&= e^{-\int_{t_i}^{t_j} \beta} \int_t^{t_i} \sigma_u^2 e^{-2 \int_u^{t_i} \beta} du \\
&= e^{-\int_{t_i}^{t_j} \beta} \gamma_{i, i}
\end{align*}

\begin{align*}
\gamma_{i+1} &:= \gamma_{i+1, i+1} \\
&= \int_t^{t_{i+1}} \sigma_u^2 e^{-2 \int_u^{t_{i+1}} \beta} du \\
&= e^{-2 \beta_i (t_{i+1} - t_i)} \gamma_i + \int_{t_i}^{t_{i+1}} \sigma_i^2 e^{-2 \beta_i(t_{i+1} - u)} du \\
&= e^{-2 \beta_i (t_{i+1} - t_i)} \gamma_i + \frac{\sigma_i^2}{2 \beta_i} (1 - e^{-2 \beta_i(t_{i+1} - t_i)})
\end{align*}

\begin{align*}
\alpha_i &:=
<K_i, \int_{t_0}^{t_f} \frac{\sigma_u}{\beta_u} dW_u> \\
&=
< \int_{t}^{t_i}\sigma_u e^{-\int_u^{t_i} \beta} dW_u,
\int_{t_0}^{t_f} \frac{\sigma_u}{\beta_u} dW_u > \\
&=   \int_t^{t_i} \frac{\sigma_u^2}{\beta_u} e^{-\int_u^{t_i} \beta} du \\
\end{align*}

\begin{align*}
\alpha_{i+1}
&= e^{-\beta_i (t_{i+1} - t_i)} \alpha_i + \int_{t_i}^{t_{i+1}} \frac{\sigma_u^2}{\beta_u} e^{-\beta_i(t_{i+1} - u)} du\\
&= e^{-\beta_i (t_{i+1} - t_i)} \alpha_i + (\frac{\sigma_i}{\beta_i})^2 (1 - e^{-\beta_i(t_{i+1} - t_i)})
\end{align*}


\begin{align*}
\omega &:=
<\int_{t_0}^{t_f} \frac{\sigma_u}{\beta_u} dW_u, \int_{t_0}^{t_f} \frac{\sigma_u}{\beta_u} dW_u>\\
&= \int_{t_0}^{t_f} (\frac{\sigma_u}{\beta_u})^2 du \\
&= \sum_i  (\frac{\sigma_i}{\beta_i})^2 (t_{i+1} - t_i)
\end{align*}


\newpage

\subsection*{Le modèle à deux facteurs}
Nous revenons au modèle original à deux facteurs. Les paramètre relatifs au facteur $x$ (resp. $y$) seront dénoté par un x (resp. y) en exposant.

L'espérance étant linéaire, et la variance quadratique, 
$M(t_0, t_f)$ est remplacée par $M^x x + M^y y$, et $V(t_0, t_f)$ par $V^{xx} + V^{yy} + 2 V^{xy}$, de sorte que:

$$P(t_0, T_f) = exp(-\int_{t_0}^{t_f} \Phi - M^x(t_0, t_f) x(t_0) - M^y(t_0, t_f) y(t_0) + \frac{V(t_0, t_f)}{2})$$

\begin{align*} V^{xy} &:= <
\int_{t_0}^{t_f} \frac{\sigma^x_u}{\beta_u^x} dW_u^x+\sum_{i=1..n}c_i^x K_i^x,
\int_{t_0}^{t_f} \frac{\sigma^y_u}{\beta_u^y} dW_u^y+\sum_{i=1..n}c_i^y K_i^y> \\
&= \int_{t_0}^{t_f} \frac{\sigma_u^x \sigma_u^y}{\beta_u^x \beta_u^y} \rho_u du
+ \sum_{ij} c_i^x c_j^y <K_i^x, K_i^y>
+ \sum_i c_i^x <K_i^x \int_{t_0}^{t_f} \frac{\sigma^y_u}{\beta_u^y} dW_u^y> + c_i^y <K_i^y \int_{t_0}^{t_f} \frac{\sigma^x_u}{\beta_u^x} dW_u^x>\\
&= \omega^{x,y} + \sum_{ij} c_i^x c_j^y \gamma_{ij}^{xy}
+ \sum_i c_i^x \alpha_i^x + c_i^y \alpha_i^y
\end{align*}

avec comme pour le cas à un seul facteur:
$$\alpha^x_{i+1} = e^{-\beta^x_i(t_{i+1} - t_i)} \alpha^x_i + \rho_i \frac{\sigma^x_i \sigma^y_i}{\beta^x_i \beta^y_i} (1 - e^{-\beta^x_i(t_{i+1} - t_i)})$$
$$\gamma_{i+1} = e^{- (\beta^x_i+\beta^y_i) (t_{i+1} - t_i)} \gamma_i + \rho_i \frac{\sigma_i^x \sigma_i^y}{\beta_i^x + \beta_i^y} (1 - e^{- (\beta^x_i+\beta^y_i)(t_{i+1} - t_i)})$$
$$\omega := \sum_i \rho_i  \frac{\sigma_i^x \sigma_i^y}{\beta_i^x \beta_i^y} (t_{i+1} - t_i)$$

\iffalse
With $i \leq j$:
\begin{align*}
 \gamma_{ij}^{xy} &:= <K_i^x, K_i^y> \\
  &= <\int_t^{t_i} \sigma_u^x e^{-\int_t^{t_i} \beta^x} dW^x, \int_t^{t_j} \sigma_u^y e^{-\int_t^{t_j} \beta^y} dW^y> \\
  &= e^{-\int_{t_i}^{t_j} \beta_y} \int_t^{t_i} \sigma_u^x \sigma_u^y e^{-\int_t^{t_i} \beta^x + \beta_y} \rho_u du \\
&= e^{-\int_{t_i}^{t_j} \beta_y} \gamma_{ii}
\end{align*}
\fi


