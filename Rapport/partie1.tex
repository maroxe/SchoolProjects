%%% Local Variables:
%%% mode: latex
%%% TeX-master: "main"
%%% End:


\chapter{Préliminaires sur les taux d'interêts}

\section{Définiton}
\begin{defn}
On dénote le prix de zéro coupon forwardé $P(T, S)$ le montant qu'il faut investir à dans un instrument risque-neutre au temps $T$ pour obtenir une unité de monnaie au temps $S$.
\end{defn}

\begin{defn}
On définit $f(t, T)$ le taux d'intêret instantané forwad à la date $t$ pour une maturité $T$ la quantité $$f(t, T) := - \frac{ \delta}{\delta T}  log P(T, S)$$
\end{defn}

\begin{defn} Le taux instantanné est défini par
  $r_t = \underset{T \to t}{lim}f(t, T) $ \\
  Le taux d'actualisation (stochastique) est: $D_t^T := e^{-\int_t^Tr_s \rm{d}s}$
\end{defn}

\iffalse
\begin{defn}
  Le taux d'interet cumulé entre deux période $t$ et $T$ est la quantité $R(t, T)$ que $r_t$ doit égaler spour avoir le même rendement
\end{defn}
\fi

\begin{defn}
  Le taux d'interet forward $F(t; T, S)$ est la prévision  à l'instant $t$ du taux entre deux période $T$ et $S$.
  Ce taux est la quantité $L$ connu à l'instant qui annule la valeur de ce contrat à l'instant $t$:
  \begin{itemize}
  \item Recevoir l'interêt  $L$ sur un 1 euro entre $T$ et $S$
  \item Payer le taux variable  $F(T, S)$ sur un 1 euro entre $T$ et $S$
  \end{itemize}
\end{defn}


le taux $r_t$ n’est pas un produit échangé sur le marché que l’on peut mettre en portefeuille. On ne peut donc pas construire de couverture d’un produit donné de la même manière que dans un modèle d’action, et ce malgré la similitude des modèles mathématiques.

\section{Mesures équivalentes}

Nous nous plaçons dans le cadre d'une économie à temps continu, qui admet une espace de probabilité $(\Omega, \cal{F}, \mathbb{P})$, avec $K+1$ actifs tradables, que nous appelerons actifs de base, dont le prix est donné par $(S_t = (S^0_t, ...S^k_t))_{t \geq 0}$. Dans toute la suite nous confondons l'actif et son prix.
$S^0$ étant l'actif sans risque, qui évolue donc au temps sans risque $$\mathrm{d}S^0_t = r_t S^0_t \mathrm{d}t$$
ie $$S^0_t = e^{-\int_0^t r_s \mathrm{d}s}$$

Par définition, nous connaissons le prix des actifs $K+1$, dans la prochaine section nous détaillerons la procédure de pricing de produits plus compliqués.

\subsubsection{Principe de pricing}
A travers les $K+1$ actifs de base, nous construisons des produits plus complexes. 
Le prix de ce dernier est donc intimement lié à la possibilité de trouver une stratégie auto-financée qui le réplique.
Commençons d'abord par définir ce qu'est une stratégie auto financé.

\begin{defn}
  \begin{itemize}
  \item Une stratégie est une processus $(\Phi_t = (\Phi^0_t, ... \Phi^K_t))_t$ localement borné et adapté à la filtration $\cal{F}$.
  \item La valeur associé à cette stratégie est donné par $V_t(\Phi) = <\Phi_t, S_t>$.
  \item Une stratégie est auto financée sir $\mathrm{d}V_t = \Phi_t \mathrm{d}S_t$
  \end{itemize}
\end{defn}

Une hypothèse souvent utilisée dans le cadre de la finance de marché est l'abscence d'abitrage. Une opportunité d'arbitrage est la possiblité d'investir 0 aujourd'hui, et recevoir, avec probabilité non nulle, un montant positive dans le future. En d'autres termes, l'abscence d'arbitrage signifie que si $\Phi$ est une stratégie auto financée telle que $V_0(\Phi) = 0$, alors $\mathbb{P} ( V_t(\Phi) > 0 ) = 0$. Ceci nous permettra de valoriser des produits complexes en répliquant leur payoff par une combinaison linéaire de produits simple dont le prix est connue.

Une deuxième hypothèse que nous admettrons dans la suite est la complétude du marché: Tout les produits utilisés seront consiédérés disponible à tout moment et en quantité abandante (liquide), ie à chaque instant $t$, pour tout payoff $H$, il existe une stratégie autofinancée associée $\Phi$ qui vérifie $V_t( \Phi ) = H$. Nous ne traiterons pas le cas des produits illiquide. Ceci est justifié, le marché des taux étant l'un des plus gros en volume dans le monde.

Nous pouvons montrer(*) que ces hypothèse sont équivalent à l'existence d'une mesure de probabilité risque neutre $Q$ unique sous laquelle le prix actualisé de tous les produits tradables sont des martingales. ie si un on note $H_t$ le prix à l'instant $t$ d'un produit financier, alors
$$H_t =\espr{ \frac{H_s}{S^0_t} | F_t } = V_t(\Phi)$$

En particuler, le prix d'un zéro coupon qui paye 1 à l'instant $T$ est donné par
$$ P(t, T) := \espr{  e^{-\int_t^T r} } $$


Nous pouvons interpréter le ratio $\frac{H_s}{ S^0_s}$ comme étant le nombre  de $H$ par unité de discount factor stochastique $S^0$. Le discount factor est appelé dans ce cas numéraire. Nous verrons maintenant que nous pouvons choisir un autre numéraire plus adapté au produit que nous voulons pricer.
En effet le changement de numéraire préserve la propriété d'autofinancement d'un portefeuille.

\begin{defn} Un numéraire est tout actif financier ne payant pas de dividendes \end{defn}

\begin{defn} Mesure de probabilité équivalente.
  
Supposons qu’il existe un numéraire $(M_t )_{t \geq 0}$ et une mesure martingale équivalente $Q^M$ telle que le prix de chaque actif actualisé par le processus M soit une$Q_M$-martingale. 
$$ \frac{S^i_t}{M_t} = \Qespr{Q^M}{ \frac{S^i_T}{M_T} | F_t}$$
Soit $(N_t )_{t \geq 0}$ un numéraire. \\
Alors il existe une mesure de probabilité $Q_N$ telle que le prix de chaque actif actualisé par le processus $N$ soit une $Q_N$-martingale, i.e.
$$ \frac{S^i_t}{N_t} = \Qespr{Q^N}{ \frac{S^i_T}{N_T} | F_t}$$

où $Q^N$ est définie par:
$$\Qespr{Q^N}{ H } = \Qespr{Q^M}{ \frac{ M_T/N_T}{M_0/N_0} H}$$

\end{defn}

Example: Mesure forward neutre

Le bond zéron coupon dont la maturité concide avec la date du payment d'un produit financier peut servir de numéraire. Nous appellerons la mesure de probabilité associé $Q_T$.

Dans ce cas $P(T, T) = 1$, et par conséquent il suffit de calculer l'espérance du payoff (divisé par 1) sous $Q_T$.
Si nous notons le payoff de ce produit $H$, alors son prix à l'instant $0$ est donné par $$P(t, T) \, \Qespr{Q_T}{ H | F_t } $$
Pour que celà nous soit utile, il faut que la dynamique de $H$ soit connue sous $Q_T$. Ceci est vérifié pour les contrats payant un taux d'interêt sur un nominal fixe. En effet $(F(t; S, T))_t$ est une martingale 
$$ \Qespr{Q_T}{ F(t; S, T) | F_u } = F(u; S, T)$$

\textbf{Preuve:}
Si nous disposons de  $\frac{P(t, S)}{(1+(T-S)F(t; S, T))}$ au temps $t$, nous pouvons acheter $\frac{1}{(1+(T-S)F(t; S, T))}$ unités de l'obiligation $P(t, S)$, nous obtenons $\frac{1}{(1+(T-S)F(t; S, T))}$ au temps $S$, cette somme là est, par définition de $F$, équivalente à l'obtention de 1 à l'instant $T$, qui exactement le payoff de l'obligation $P(t, T)$.
Par principe de \textbf{non arbitrage}, ces deux investissement doivent avoir le même coût, ie:
$ \frac{1}{T-S} \left( \frac{P(t, S)}{P(t, T)} - 1  \right) $
La preuve en découle.

\newpage

\section{Produits financier d'interêts}

Le dévelopement de la section précédente nous sera utile pour pricer les dériver des taux.
Considérons le cas particlier d'un call eurpéen à maturité $T$, strike $K$, dont le sous-jacent est bond zéro coupon qui expire à l'instant $S$. Le payoff d'un tel contrat est connu: $ (P(T, S) - K)^+)$. Son prix à un instant antérieur $t$ est
$$ZBC(t, T, S, K) := \espr{ e^{-\int_t^T r_s \rm{d}s} \, (P(T, S) - K)^+ | F_t }$$
Il est plus pratique de considérer la forward mesure, sous laquelle le prix du call s'écrit
$$ZBC(t, T, S, K) = P(t, T) \, \Qespr{Q_T}{(P(T, S) - K)^+ | F_t}$$
De même, pour un put
$$ZBP(t, T, S, K) = P(t, T) \, \Qespr{Q_T}{(K - P(T, S))^+ | F_t}$$

Cette écriture nous rappelle la formule de blackscholes pour les options sur les actions.

\begin{defn}
  Swap:
Un swap est un contract entre deux parties qui s'engagent à échanger des flux financiers pendant une durée et à une fréquence détérminées. la plupart du temps, ces flux sont détérminé comme étant l'intêret à un taux fixe $K$ contre un taux variable (LIBOR ${L(T_i)}_i$ par examplesur un notionnel $N$. 
$$ N \sum D_t^{T_i} \tau_i (L(T_{i-1}) - K) $$
\end{defn}


\begin{defn}
  Caplet/floorlet:
  Un caplet/floorlet peut être vu comme un call/put européenne sur un
  Son payoff est le suivant
$$ \tau (L(T, S) - K)^+ $$
\end{defn}

\begin{align}
  Cpl(t, T, S, \tau, X)
  &= \espr{ e^{-\int_t^S r_s \rm{d}s} \tau (L(T, S) - K)^+ | F_t} \\ 
  &= \espr{ e^{-\int_T^S r_s \rm{d}s} P(t, T)  \tau (L(T, S) - K)^+ | F_t} \\
  &= \espr{ e^{-\int_T^S r_s \rm{d}s} (1 - (1 + X \tau)P(t, T))^+ | F_t} \\
  &= (1 + X \tau) \espr{ e^{-\int_T^S r_s \rm{d}s} (\frac{1}{1 + X \tau} - P(t, T))^+ | F_t} \\
  &= (1+X \tau) ZBP(t, T, S, \frac{1}{1+X \tau})
\end{align}

\begin{defn}
  Cap/floor:
Un cap/floorlet peut être vu comme une somme de caplets
$$ N \sum D_t^{T_i} \tau_i (L(T_{i-1}) - K)^+ $$
\end{defn}

La forme des payoff indique que le cap permet de protéger son détenteur d’une hausse des taux Libor, et symétriquement que le floor protège d’une éventuelle baisse de ces taux.


%%% Local Variables:
%%% mode: latex
%%% TeX-master: "main"
%%% End:
