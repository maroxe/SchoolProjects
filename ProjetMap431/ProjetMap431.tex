\documentclass{article}

\usepackage[utf8]{inputenc}  
\usepackage[T1]{fontenc}   
\usepackage{amsmath}
\usepackage{amsfonts}
\usepackage{pgfplots}
\usepackage{sagetex}
\usepackage{graphicx}
\usepackage{caption}
\usepackage{subcaption}
\usepackage{minted}

\usepackage{geometry}
\geometry{hmargin=1.5cm,vmargin=1.5cm}

\newcommand{\BigO}[1]{\ensuremath{\operatorname{O}\left(#1\right)}}

\newcommand{\Deriv}[1]{\ensuremath{\frac{d}{dx}(#1)}}
\newcommand{\DDeriv}[1]{\ensuremath{\frac{d#1}{dx}}}
\newcommand{\SmallO}[1]{\ensuremath{\operatorname{o}\left(#1\right)}}

\newcommand{\BBigO}[3]{\ensuremath{\underset{#1 \to #2 }{\operatorname{O}\left(#3\right)}}}
\newcommand{\BigF}[2]{\ensuremath{#1 \left(#2\right)}}
\newcommand{\Wrap}[1]{\ensuremath{\left(#1\right)}}
\newcommand{\Q}[1]{\subsubsection*{Question #1}}

\newcommand{\Code}[2]{\textbf{#1}: #2\\}
\newcommand{\Source}[0]{\subsubsection*{Code sources:}}


\begin{document}

\title{MAP431 - Projet}
\author{Anas BIAD  - Bachir EL KHADIR }


\maketitle

\Q{1} 

Soit $ v \in H_{0}^{1}(\Omega) $ \\
On multiplie l'équation par la fonction test $v$ et on intègre par parties en utilisant la formule de Green : 
\begin{align*} 
 \int_{\Omega}  f(x)v(x)dx   &=  \int_{\Omega}  v(x)u_{\mu}(x)dx   -  \int_{\Omega}  \Deriv{ \kappa_{\mu}(x) \DDeriv{u_{\mu}}(x) } v(x)dx  \\
&= \int_{\Omega}  v(x)u_{\mu}(x)dx  +  \int_{\Omega}  \kappa_{\mu}(x) \DDeriv{v}(x) \DDeriv{u_{\mu}}(x) dx \\
\end{align*}
\\
Donc la forumation variationnelle du problème (3) s'écrit : 
$$a_{\mu}(u_{\mu},v) = b(v)$$ 
où : 
$$a_{\mu}(u_{\mu},v) = \int_{\Omega}  v(x)u_{\mu}(x)dx  +   \int_{\Omega}  \kappa_{\mu}(x) \DDeriv{v}(x) \DDeriv{u_{\mu}}(x) dx $$ 
$$b(v) =  \int_{\Omega}  f(x)v(x)dx $$

% -------------------------------------------- QUESTION 2 --------------------------------------------
\Q{2}

Les coefficients de A:
\begin{align*}
A_{i,j} = a_{\mu}(\varphi_{i},\varphi_{j}) =
\left\{\begin{array}{ll}
\frac{2}{h^{2}} \int_{K_{i}} \kappa_{\mu}(x) dx + \frac{2}{3}h & \mbox{si} \, i = j \\
-\frac{1}{h^{2}} \int_{K_{i}} \kappa_{\mu}(x) dx + \frac{h}{4} & \mbox{si} \, j = i+1 \\
-\frac{1}{h^{2}} \int_{K_{i-1}} \kappa_{\mu}(x) dx + \frac{h}{4} & \mbox{si} \, j = i-1 \\
0 & \mbox{sinon} \\
\end{array}
\right.
\end{align*}

Les coefficients de B:
\begin{align*}
B_i = b(\varphi_{i}) &= \int_{0}^{1} \varphi_{i}(x) sin(2\pi x)dx  \\
&= \int_{0}^{1} \varphi(\frac{x-x_{i}}{h}) sin(2\pi x)dx \\
&= h \, sin(2 \pi x_{i}) \,  (sinc(\pi h))^{2}
\end{align*}


%---------------------------------------   QUESTION 3 --------------------------------
\Q{3}

\Source
\Code{calcul.sce}{contient les fonctions calculant A, B puis la solution X}
\Code{question3.sce}{calcule et affiche le résultat pour $\mu \in \{0.01, 0.1, 1\}$}


\subsubsection*{courbe de $u_\mu$ pour $\mu \in \{0.01, 0.1, 1\}$}
\includegraphics[scale=0.25]{img/q3.png}



\subsubsection*{Commentaire}
\begin{itemize}
	\item Les profils pour $\mu \in \{0.01, 0.1\}$ ne varient que dans la partie où $\kappa$ est différent.
	\item On remarque une discontinuité brusque de la tendance des courbes lors du passage de $\Omega_1$ à $\Omega_2$.
	\item Le calcul prend beaucoup de temps.
\end{itemize}



% ---------------------------------------   QUESTION 4 --------------------------------------------
\Q{4}
En injectant la relation:
$$ u_{\mu}^{RB} = \sum_{j=1}^{N_{0}} X_{\mu , j }^{RB} u_{j} $$ 
Dans:
$$ a_{\mu}( u_{\mu}^{RB} , u_{i} ) = b(u_{i}) $$ et 
On trouve : 
$$ \sum_{j=1}^{N_{0}} X_{\mu , j }^{RB} a_{\mu}(u_{j},u_{i}) = b(u_{i}) $$ 

Ainsi $ X_{\mu}^{RB} $ est solution du système linéaire : 
$$ A_{\mu}^{RB}X_{\mu}^{RB} = B^{RB} $$ 
Avec : 
$$ (A_{\mu}^{RB})_{i,j} = a_{\mu}(u_{j},u_{i}) $$
et : 
$$ (B^{RB})_{i} = b(u_{i}) $$

On a : 
\begin{itemize}
	\item $a_{\mu}$ est symétrique 
	\item $a_\mu$ est positive car $(\forall v \in H^0_1(\Omega)) \, a_{\mu}(v,v) = \int_{\Omega}  (v(x))^{2}dx  +   \int_{\Omega}  \kappa_{\mu}(x) (\DDeriv{v}(x))^{2}  dx >= 0$ 
	\item Si $  a_{\mu}(v,v) = 0 $ alors $ \int_{\Omega}  (v(x))^{2}dx = 0 $ , donc $v=0$ . Ainsi $ a_{\mu} $ est définie 
\end{itemize}
$ A_{\mu}^{RB} $ est la matrice de $a_\mu$ dans la base des $u_i$, donc $ A_{\mu}^{RB} $ est symétrique , définie positive.


%--------------------- ----------------------   QUESTION 5 ----------------------------------------
\Q{5}
on a : 
\begin{align*}
(A_{\mu})_{i,j} &= a_{\mu}(u_{\mu},v) \\
&= \int_{\Omega}  \varphi_i(x) \varphi_j(x)dx  +   \int_{\Omega_{1}} \DDeriv{\varphi_i}(x) \DDeriv{\varphi_j}(x) dx  +  \mu \int_{\Omega_{2}} \DDeriv{\varphi_i}(x) \DDeriv{\varphi_j}(x) dx \\
&= \int_{\Omega} \varphi_{j}(x) \varphi_{i}(x)dx  +   \int_{\Omega_{1}} \DDeriv{\varphi_{j}}(x) \DDeriv{\varphi_{i}}(x) dx 
+ \int_{\Omega_{2}} \DDeriv{\varphi_{j}}(x) \DDeriv{\varphi_{i}}(x) dx \\
&= (A_{0})_{i,j} + (A_{1})_{i,j}
\end{align*}
Où l'on a posé:
$$ A_{0} = \left(\int_{\Omega} \varphi_{j}(x) \varphi_{i}(x)dx  +   \int_{\Omega_{1}} \DDeriv{\varphi_{j}}(x) \DDeriv{\varphi_{i}}(x) dx\right)_{i, j} $$
$$ A_{1} = \left(\int_{\Omega_{2}} \DDeriv{\varphi_{j}}(x) \DDeriv{\varphi_{i}}(x) dx\right)_{i, j} $$
Et donc on a: $$ A_{\mu} = A_{0} + \mu A_{1} $$


D'autre part , on a 
\begin{align*}
(A_{\mu}^{RB})_{i,j} &= a_{\mu}(u_{j},u_{i}) \\
&= [u_{i}]_{(\varphi_{1},...,\varphi_{N})}^T \, A_\mu \, [u_{j}]_{(\varphi_{1},..,\varphi_{N})} \\
&= X_i^T A_\mu X_j \\
&= X_i^T A_0 X_j + \mu X_i^T A_1 X_j \\
&= (A_0^{RB})_{i,j} + \mu (A_1^{RB})_{i,j}
\end{align*}
et :
$$ (B^{RB})_{i} = b(u_{i}) = B^T X_i $$


%--------------------------------------------- QUESTION 6 ----------------------------
\Q{6}

\Source
\Code{calcul2.sce}{qui permet de calculer la solution avec la nouvelle méthode}
\Code{question6.sce}{calcule et affiche les solutions}

\begin{center}
\includegraphics[scale=0.25]{img/q6-1.png}

Graphe de $u_\mu$ et $u_{rb}$
\end{center}

Les deux solutions sont quasiment confondues.

\begin{center}
\includegraphics[scale=0.25]{img/q6-2.png}

Graphe de $|u_\mu -u_{rb}|$
\end{center}



L'erreur est de l'ordre de $10^{-4}$. La nouvelle méthode en plus d'être efficace, fournit des solutions acceptables.

%---------------------------------------------   QUESTION 7 -----------------------
\Q{7}
\Source
\Code{question7.sce}{applique la méthode de l'énoncé}

\begin{center}
\includegraphics[scale=0.25]{img/q7-1.png}

$\mu$ uniforme sur $[1.1, 1]$
\end{center}
\begin{itemize}
	\item Les deux solutions sont assez différentes. 
	\item La nouvelle de méthode de résolution de Galerkin permet de simuler un grand nombre de solutions en un temps raisonnable.
\end{itemize}
\begin{center}
\includegraphics[scale=0.25]{img/q7-2.png}

$\mu$ de loi log-uniforme
\end{center}
Les deux solutions sont proche dans ce cas de figure. 



%---------------------------------------------   QUESTION 8 -----------------------
\Q{8}
\Source
\Code{equation.edp}{implémente l'algorithme sur FreeFem++ }

\subsubsection*{Résultat graphique}
\begin{center}
\begin{tabular}{cc}

  \includegraphics[scale=0.5]{img/q8-1.png}	
  &
  \includegraphics[scale=0.5]{img/q8-2.png}	
	\\
	
	$\mu = 0.01$
	 & 
	 $\mu = 0.075$
	\\
  \includegraphics[scale=0.5]{img/q8-3.png}	
  &
  \includegraphics[scale=0.5]{img/q8-4.png}	
	\\
	
	$\mu = 0.1$
	 & 
	 $\mu = 1$
	\end{tabular}
\end{center}
\end{document}