%%% Preamble
\documentclass[paper=a4, fontsize=11pt]{scrartcl}
\usepackage[utf8]{inputenc}  
\usepackage[T1]{fontenc}
\usepackage{fourier}
\usepackage[english]{babel}															% English language/hyphenation
\usepackage[protrusion=true,expansion=true]{microtype}	
\usepackage{amsmath,amsfonts,amsthm} % Math packages
\usepackage[pdftex]{graphicx}	
\usepackage{url}

%%% Custom sectioning
\usepackage{sectsty}
\allsectionsfont{\centering \normalfont\scshape}

%%% Custom headers/footers (fancyhdr package)
\usepackage{fancyhdr}
\pagestyle{fancyplain}
\fancyhead{}											% No page header
\fancyfoot[L]{}											% Empty 
\fancyfoot[C]{}											% Empty
\fancyfoot[R]{\thepage}									% Pagenumbering
\renewcommand{\headrulewidth}{0pt}			% Remove header underlines
\renewcommand{\footrulewidth}{0pt}				% Remove footer underlines
\setlength{\headheight}{13.6pt}


%%% Equation and float numbering
\numberwithin{equation}{section}		% Equationnumbering: section.eq#
\numberwithin{figure}{section}			% Figurenumbering: section.fig#
\numberwithin{table}{section}				% Tablenumbering: section.tab#


%%% Maketitle metadata
\newcommand{\horrule}[1]{\rule{\linewidth}{#1}} 	% Horizontal rule

\title{
		%\vspace{-1in} 	
		\usefont{OT1}{bch}{b}{n}
		\normalfont \normalsize \textsc{Ecole Polytechnique} \\ [25pt]
		\horrule{0.5pt} \\[0.4cm]
		\huge Modèles des Taux d'interêt \\
		\horrule{2pt} \\[0.5cm]
}
\author{
		\normalfont \normalsize
                Bachir EL KHADIR\\[-3pt] \normalsize
                \today	
}
\date{}

\newtheorem{theorem}{Theorem}

\newcommand{\IMG}[3]{
\includegraphics[scale=#3]{#1}%
}


%%% Begin document
\begin{document}
\maketitle

\newpage
\tableofcontents

\newpage

%%% Local Variables:
%%% mode: latex
%%% TeX-master: "main.tex"
%%% End:

\chapter*{Remerciments}

Je tiens à remercier toutes les personnes qui ont contribué au succès de mon stage et qui m'ont aidé lors de la rédaction de ce rapport.

Tout d'abord, j'adresse mes remerciements à mon professeur, Mr Stefano De Marco de l'Ecole Polytechnique qui m'a beaucoup aidé dans ma recherche de stage et ma permis de postuler dans cette entreprise. Son écoute et ses conseils m'ont permis de cibler mes candidatures et de trouver ce stage me correspondait totalement.

Je tiens à remercier vivement mon maitre de stage, Mr John Paul Barjaktarevic, responsable du service Y au sein de la banque JP Morgan, pour son accueil, le temps passé ensemble et le partage de son expertise au quotidien. Grâce aussi à sa confiance j'ai pu m'accomplir totalement dans mes missions avec son aide précieuse dans les moments les plus délicats.

Je remercie également toute l'équipe ``rates exotic'' pour leur accueil, leur esprit d'équipe. 

Enfin, je tiens à remercier toutes les personnes qui m'ont conseillé et relu lors de la rédaction de ce rapport de stage : ma famille, mon amie Julie B camarade de promotion.


\newpage

%%% Local Variables:
%%% mode: latex
%%% TeX-master: "main"
%%% End:

\newpage

\subsection*{Résumé}
  
  Ce rapport présente quelques outils utilisés lors du pricing de dérivés de taux. 
  Il est particulièrement centré sur le modèle de Hull White qui est conduit par deux facteurs (appelé aussi G2++).  J'adopterai le plan suivant dans la rédaction de ce rapport:
  \begin{itemize}
  \item 
    La première partie introduit le marché de taux par un certain nombre de définitions qui lui sont propores, ainsi que les produits financiers principaux utilisés pour la calibration des modèles.
  \item 
    Dans la deuxième partie je présenterai quelques modèles existant, et je discuterai de leurs limites et de leurs implémentation pratique.
  
  \item
    La troisième partie présente les résultats concrets que j'ai obtenus dans le cas du modèle gaussien à deux facteurs, ainsi que la partie calibratoin de modèle.
  \end{itemize}
  
\subsection*{Abstract} 
This report presents some tools used in the pricing for intereset rates derivatives. It's focused on two factors Hull-White model (called G2++).  
  \begin{itemize}
  \item
    The first part introduces some definitions relative to the interest rates market, as well as some derivative used later for the calibration process.
  \item
    In the second part I will present some existing models, their limitations and possible implementation.

  \item
    The last part presents the results I obtained using the G2++ model and the calibration process.
  \end{itemize}




\newpage
\chapter*{Introduction}

J'ai fait mon stage a JP Morgan equipe exotic rates. La plus grosse partie du travail consitait à trouver le bon modèle mathématique pour pricer un produit et le calibrer au marche.

L'interêt que porte les banques aux produits exotiques a considérablement augmenté aucours des dernières années. Plusieurs modèles ont été dévelopé pour refeléter au mieux le comportement des marchés financiers.

Le département au sein duquel j'ai effectué mon stage s'occupait des produits basé sur les taux d'intêret. Dans ce domaine, une propriété appréciable dans un modèle est le fait qu'il possèd un nombre de paramètres suffisant pour être calibré parfaitement aux prix observable dans la réalité, sans pour autant ``overfitter'' l'échantillon disponible. En effet, dérnièrement, surtout depuis la crise, le marché des taux a connu des changement radicaux, la possiblité d'observer des taux négatifs en est un example. 

Un produit en particulier n'a cessé de  gagner en notoriété: les cancelables spread options. Ma mission de stage était de comprendre les modèles existant et leurs implémentations qui permettent de pricer (entre autres) ce type de produits, comprendre leurs limites, et essayer de trouver des améliorations.

Ce stage est à forte composante informatique, une attention particulière a été accordée aux détails d'implémentation et optimisation du code.
 
\newpage

La notion de taux d'interêt nous est familière et fait parti de notre vie de tous les jours.

Zéro Coupon

\section{Quelques définitions}
\begin{theorem}
$Z(t, T, S)$ est le montant qu'il faut investir dans un instrument risque-neutre au temps $T$ pour obtenir 1euro au temps S.
\end{theorem}
\begin{theorem}
$r$ le taux d'intêret instantné est défini comme étant
\end{theorem}
\begin{theorem}
La courbe de rendement ou la yield curve $L(t, T)$
\end{theorem}
\begin{theorem}
Le taux d'interet cumulé
\end{theorem}

\newpage
\section{Produits financier d'interêts}
Swap:
Un swap est un contract entre deux parties qui s'engagent à échanger des flux financier pendant une durée et à une fréquence détérminées. la plupart du temps, ces flux sont détérminé comme étant l'intêret sur un notionnel K. 
$$ N \sum D_t^{T_i} \tau_i (L(T_{i-1}) - K) $$

Caps/floor:
Un cap/floor peut être vu comme un call/put européenne sur un swap
$$ N \sum D_t^{T_i} \tau_i (L(T_{i-1}) - K)^+ $$

Swaption
$$ N (\sum D_t^{T_i} \tau_i (L(T_{i-1}) - K))^+ $$

\newpage
\section{Les différents modèles des taux d'interêts}

Certains produits financiers dépendent en grande partie de la courbe de rendement. Nous avons vu précédemment que la donnée du taux instantané $r_t$ permet de caractériser complètement cette courbe.

Il est donc important que la dynamque de $r_t$ soit à la fois riche pour pouvoir décrire la courbe de rendement observée dans le marché, et suffisemment simple pour que le temps nécessaire pour le calcul ne soit trop long.

Nous pouvons considérer que la courbe de rendement varie dans un espace vectoriel de dimension infinie. Toute tentative de la caracteriser par un nombre fini de paramètre rééls est donc vouée à l'échec.

Le modèle de Hull White a été introduit en 1990.  Un des atouts majeure de ce modèle est la possiblité de simuler la dynamique de $r_t$ par un arbre trinomiale. Ceci étant essentiel pour pricer des produits du type bermuda options
$$ \mathrm{d}r_t =  (\theta(t) - \alpha(t) r_t) \mathrm{d}t + \sigma(t) \mathrm{d} W_t$$
$$ r_t = e^{-\alpha t} r_0 + integral ...$$


\newpage
\section{Le modèle à deux facteurs}
\subsection{Motivation}



Montrons dans un premier temps pourquoi un modèle à un seul facteur n'est pas suffisant pour pricer certains produits disponible sur les marchés. En effet, considérons un produit $E$ dont le payoff  dépent  du spread entre un taux d'intêret cumulé entre $0$ et $T_1$ pour le premier et $0$ et $T_2$ pour le second.

La dynamique de $r_t$ dans le modèle de Vasicek est donné par

$$r_t = k(\theta - r_t)  \mathrm{d}t  + \sigma \mathrm{d}W_t$$
La formule analytique du bond zéro coupon est donc
$$P(t, T) = A(t, T) exp(-B(t, T) r_t)$$
En particulier le taux d'intêret cumulé est une donné par une transformation affine du taux instantané:
$$R(t, T) = \frac{ln P(t, T)}{T-t} =: a(t, T) + b(t, T) r_t$$
Le payoff du produit $E$ est donc fonction de la distribution jointe de $R(0, T_1)$ et $R(0, T_2)$. Sauf que:
$$Cor(R(0, T_1), R(0, T_2) = 1$$
On en déduit qu'un choc à $r_t$ agit de la même manière sur toutes les courbes
On observe sur les marchés que les taux à différentes maturités ne sont pas corrélés. Si on regarde le taux 2Y et 10Y

<Courbe 10Y> 

<Courbe 2Y>

Clairement un tel modèle ne capture pas ce comportement.

Dans cette section nous considérons un modèle où le taux d'intêret instantanté est donné par une somme de deux facteurs gaussiens centrés et corrélés. Pour fitter la yield courbe, on rajoute une fonction détérministe.

\begin{align*}
  \rm{d}X &= \\
  \rm{d}Y &= \\
  \rm{d} r &= X + Y + \phi(t)
\end{align*}

$$P(t, T) = E[...]$$

Le processus est markovien

Les cap floors ademettent une formule analytiques, comme pour les options vanilles : formule de blackscholes
Dans la pratique, les banques tradent beaucoup des produits exotiques. Ce n'est pas le cas pour ces produits => arbre de pricing

\subsection{Approximation de la solution par un arbre binomial}

Cette méthode a été d'abord suggéré par Hull-White (1994)
On commence par donner une approximation de la dynamique processus $x$ et $y$ :

$$E(x(t+\Delta t) | F_t) = x(t) e^{-a \Delta t}$$
$$V(x(t+\Delta t) | F_t) = \frac{\sigma^2}{2a} (1 - e^{-2a \Delta t})$$
$$Cov\{x(t+\Delta t), y(t+\Delta t) | F_t \} = \frac{\sigma \nu \rho}{a + b} (1-e^{-(a+b)\Delta t})$$

Dans l'abre, on commence par discrétiser le temps par un pas fixe $\Delta t$, ensuite à chaque pas de temps on fait l'approximation du couple aléatoire à cdf continue $(x, y)$ par un couple de variables de bernouilli qui a les même moments de premier et second ordre.

Le nombre de noeuds augmente de façon exponentielle en fonction du nombre de pas de temps. En pratique ceci est problématique et conduit vite à une saturation de mémoire. Pour palier à ce problème on  réutilise les noeuds


\IMG{img/slice.png}{Slice}{0.5}

Cependant, on ne peut ajouter un drift au noeud avec cette methode.

Paragraphe sur la taille des slice, ellipsoid
<example de Slice>


\begin{itemize}
\item Discrétisation
\item Construction de l'abre
\end{itemize}

\subsection{Formule analytique - Intégration}
Closed form des zero coupons

en pratique, gain en temps \\
< cachegrind results > \\

gain en precision \\
<Comparaison de slice entre cf et bf>


Amerlioration:
\begin{itemize}
\item Time dependant parameters
\end{itemize}

\subsection{Performance}
L'arbre est long mais beaucoup plus puissant
Imperfections de l'arbre:
\begin{itemize}
\item  bornee
\item  discretisation
\item  probabilite negative
\end{itemize}
Monte Carlo
Limitation
Closed Form

Le temps d'execution:
Arbre
Closed form

\newpage
\section{Application: calibration et pricing}
Le modèle à 2F permet de caputrer le hump de la courbe de rendement
Le nombre de paramètre est fini (5) => pas de overfitting


\subsection{Calibration}
Dans cette section on verra a quel point le modele peut fitter la realite.

\begin{itemize}
\item $h(t)$ pour reconstruire la yield curve $\sigma \rho \nu$ pour
\item  matcher la surface
\end{itemize}

\subsubsection{Calibration du drift}

$\Pi_j$ state price (arrow debrew) (paye 1 si le noeud $(t_n, j)$ est atteint.

$$D_j(h_n) := \frac{1}{1 + r(t_n)(t_{n+1} - t_n)} $$ Le discount factor
$$ \Pi_j(t_{n+1}) = \sum_j \Pi_j(t_n) p_{j, j'}(t_n) D_{j'}(h_n)$$
$$ \sum \Pi_j D_j(h_n) = P(0, t_{n+1}) $$

Developement de taylor => trouver $h_n$


\subsubsection{Methode de calibration de la surface - Méthode d'optimisation}

On calibre les caplets

Les caplets ne sont pas directement disponilbe sur les marche
On calibre les swaptions

Le probleme de calbration est un probleme d'optimisation
Courbe de vol implicite :
minimisation de l'erreur L2

\IMG{img/capsurf.png}{Cap surface}{0.5}

\end{document}

%%% Local Variables:
%%% mode: latex
%%% TeX-master: t
%%% End:
