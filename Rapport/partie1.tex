

\section{Quelques définitions}
\begin{theorem}
$Z(t, T, S)$ est le montant qu'il faut investir dans un instrument risque-neutre au temps $T$ pour obtenir 1euro au temps S.
\end{theorem}
\begin{theorem}
$r$ le taux d'intêret instantné est défini comme étant
\end{theorem}
\begin{theorem}
La courbe de rendement ou la yield curve $L(t, T)$
\end{theorem}
\begin{theorem}
Le taux d'interet cumulé
\end{theorem}

\newpage
\section{Produits financier d'interêts}
\begin{theorem}Swap:
Un swap est un contract entre deux parties qui s'engagent à échanger des flux financier pendant une durée et à une fréquence détérminées. la plupart du temps, ces flux sont détérminé comme étant l'intêret sur un notionnel K. 
$$ N \sum D_t^{T_i} \tau_i (L(T_{i-1}) - K) $$
\end{theorem}

Caps/floor:
Un cap/floor peut être vu comme un call/put européenne sur un swap
$$ N \sum D_t^{T_i} \tau_i (L(T_{i-1}) - K)^+ $$
nominal $N = 1$
$$ CPL(t, t_{i-1}, t_i, \tau_i, X) = (1+X \tau_i) ZBP(t, t_{i-1}, t_i, \frac{1}{1+X \tau_i})$$
$$ZBP = \mathbb{E}^{Q_T}(  (P(t_{i-1}, t_i) - X)^+ )$$
Si $P(t, T)$admet une loi log normal par rapport a la mesure forward neutre $Q_T$, on peut appliquer black scholes. Dans la pratique, les trader sont habitue a raisonner en terme de vol bs plutot qu'en terme de prix. Nous verrons plus loin que la dynamique $P(t, T)$ est plus complique, mais cette methode permet neamoins d'avoir une correspondance entre prix et vol.

Swaption
$$ N (\sum D_t^{T_i} \tau_i (L(T_{i-1}) - K))^+ $$

%%% Local Variables:
%%% mode: latex
%%% TeX-master: "main"
%%% End:
