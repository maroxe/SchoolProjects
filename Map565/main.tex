\documentclass{article}

\usepackage[utf8]{inputenc} 
\usepackage[T1]{fontenc}   
\usepackage{amsmath}
\usepackage{amsfonts}
\usepackage{geometry}

\newcommand{\BigO}[1]{\ensuremath{\operatorname{O}\left(#1\right)}}
\newcommand{\SmallO}[1]{\ensuremath{\operatorname{o}\left(#1\right)}}
\newcommand{\BBigO}[3]{\ensuremath{\underset{#1 \to #2 }{\operatorname{O}\left(#3\right)}}}
\newcommand{\BigF}[2]{\ensuremath{#1 \left(#2\right)}}
\newcommand{\Wrap}[1]{\ensuremath{\left(#1\right)}}
\newcommand{\Q}[1]{\subsubsection*{Question #1}}


\begin{document}

\title{MAP}
\author{EL KHADIR Bachir}


\maketitle


Let's note $\epsilon \sim WN(0, \sigma^2)$ the innovation process of $X$ so that $X_t = \phi X_{t-1} + \epsilon_t = \phi^i X_{t-i} + \sum_{k=0}^i \phi^k \epsilon_{t-k}$

\Q{1}
$X_{l+i} - \phi^{i+1}X_{l-1} = \sum_{k=0}^{i+1} \phi^k \epsilon_{l+i-k} \perp span(\epsilon_s, s < l) = span(X_s, s < l)$ \\
$var(X_{l+i} - \phi^{i+1}X_{l-1}) = var(\sum_{k=0}^{i} \phi^k \epsilon_{l+i-k}) = \sum_{k=0}^{i} \phi^{2k} \sigma^2 = \frac{1-\phi^{2(i+1)}}{1-\phi^2} \sigma^2 = (1-\phi^{2(i+1)}) var(X_t)$ \\
if we note $Y_t = X_{-t}$, then $Y$ has the same second order properties as $X$ if we replace $\phi$ by $\frac{1}{\phi}$.

\Q{2}
From question 1 since $proj$ is linear, for every $Z \in span(X_s, s \leq l-1)$ we have $proj(Z | span(X_r, r \geq l-1) = proj(Z | X_{l-1})$.
Therefore $W \perp proj(X_{l-1+i}, i \geq 0)$.

For the second part of the question, we use the process $Y$ instead of $X$.

\Q{3}
There exist $$U, V \in span(X_p, X_{l-1}), span(X_F, X_{l+1})$$
such that $$proj(X|span(X_p, X_{l-1}, X_{l+1}, X_F) = U + V$$
By question 2 there exist $W_1, W_2 \perp span(X_{l-1}, X_l, X_{l+1})$ so that:
$$U = span(U|span(W_{l-1}) + W_1$$ and
$$V = span(V|span(W_{l+1}) + W_2$$ .

but $proj(X_l|span(X_p, X_{l-1}, X_{l+1}, X_F) - proj(U|X_{l-1}) - proj(V|X_{l+1}) = W_1 + W_2 \in and \perp span(X_{l-1}, X_l, X_{l+1})$

therefore this quantity is zero, and


\begin{align*}
  proj(X_l | X_{l-1}, X_{l+1}) &= proj(proj(X_l| X_p, X_{l-1}, X_{l+1}, X_F) | X_{l-1}, X_{l+1}) \\
&= proj(X_l| X_p, X_{l-1}, X_{l+1}, X_F)
\end{align*}
  
\Q{4}
$proj(X_{l+1}|X_{l-1}) = \frac{\phi^2}{1-\phi^2} = \gamma(1)$
$span(X_{l-1}) \perp span(X_{l+1}-proj(X_{l+1}|X_{l-1}))$.
\begin{align*}
  proj(X_l|X_{l-1}, X_{l_1}) &= (X_l, X_{l-1}) X_{l-1} + (X_l, X_{l+1}-\gamma(1)X_{l-1})  X_{l+1} \\
                             &= \gamma(1) (X_{l-1} + X_{l+1}) \\
                             &= \frac{\phi}{1+\phi^2}(X_{l-1} + X_{l+1})\\
\end{align*}
$$var(X_l - proj(X_l|X_{l-1}, X_{l_1})) = $$
*** WHY it's better ***
\Q{5}
$$\epsilon_l = X_l - \phi X_{l-1}$$
$$\epsilon_{l+1} = X_{l+1} - \phi X_l$$
so
$$\Phi = 
\left(
  \begin{array}{ccc}
    -\phi & 1 & 0 \\
    0 & -\phi & 1 \\
  \end{array}
  \right)
$$



\Q{6}
$||\Phi X||^2 = ||\ (Phi_B : \Phi_A) (X_B , X_A)^T + \Phi_M X_M||^2$

At the minimum
\begin{align*}
  0 = \delta ||\Phi_X||^2 &\Rightarrow (\Phi_B : \Phi_A) (X_B : X_A)^T + \Phi_M X_M = 0 \\
  &\Rightarrow X_M = -\Phi_M^T \Phi_M (\Phi_B : \Phi_A) (X_B : X_A)^T
\end{align*}

\end{document}

%%% Local Variables:
%%% mode: latex
%%% TeX-master: t
%%% End:
