\documentclass{beamer}

\usepackage{changepage}
\usepackage[utf8x]{inputenc}
\usepackage{tikz}

\mode<presentation> {

\usetheme{Copenhagen}
\usecolortheme{dolphin}

}
\usepackage{graphicx} 
\usepackage{booktabs}

\newcommand{\espr}[1]{
  \mathrm{E}^Q \left[ #1 \right]
}

\newcommand{\Qespr}[2]{
  \mathrm{E}^{#1} \left[ #2 \right]
}

\newcommand{\IMG}[3]{
  \begin{figure}[H]
    \centering
    \includegraphics[scale=#3]{#1}%
    \caption{#2}
    \label{#1}
  \end{figure}
}


\title[Calibration des modèles de taux d'interêt]{Calibration des modèles des taux d'interêt} 

\author{Bachir El Khadir} 
\institute[Ecole Polytechnique]
{
Ecole Polytechnique \\
\medskip
\textit{bachir.el-khadir@polytechnique.edu}
}
\date{\today}
\titlegraphic{\includegraphics[height=1cm]{img/logox.png}}
\begin{document}

\begin{frame}
\titlepage % Print the title page as the first slide
\end{frame}



\section{Présentation du stage}
\begin{frame}
\frametitle{Stage à JP Morgan}
\begin{itemize}
\item Équipe: exotic rates
\item Marché des taux: important en volume, liquide.
\item Particularités: Infinité de produits, impossibilité de trader $r_t$
\end{itemize}
\end{frame}

\begin{frame}
\frametitle{Plan de la présentation} 
\tableofcontents 
\end{frame}

\section{Le cadre de travail}
\subsection{Hypothèses}
\begin{frame}
\frametitle{Hypothèses de travail}
\begin{itemize}
\item Marché complet
\item absence d'arbitrage
\item $\rightarrow$ existence de stratégie de réplication pour  tous les payoff
\end{itemize}
\end{frame}

\begin{frame}
  \frametitle{Idée de la preuve}
  \only<1>{%%% Local Variables:
%%% mode: latex
%%% TeX-master: t
%%% End:

% Define styles for bags and leafs
\tikzstyle{bag} = [text width=4em, text centered]
\tikzstyle{end} = []
\begin{figure}[H]
  \centering
\begin{tikzpicture}[sloped]
  \node (a) at ( 2,2) [bag] {$1$};
  \node (b) at ( 6,1) [bag] {$u$};
  \node (c) at ( 6,3) [bag] {$d$};
  \node (d) at ( 2,4) [bag] {$1$};
  \node (e) at ( 6,4) [bag] {$1+R $};
  \draw [->] (a) to node [below] {$p^Q$} (b);
  \draw [->] (a) to node [above] {$1-p^Q$} (c);
  \draw [->] (d) to node [above] {$$} (e);
\end{tikzpicture}
\caption{Marché binomial}
\label{oldtree}
\end{figure}

}
  \only<2> {
    
    Stratégie:
    \begin{itemize}
    \item Payoff: $x$ dans la réalisation $\omega_x$, $y$ dans la réalisation $\omega_y$ 
    \item coût: $\Pi$
    \item allocation: $\alpha$ actifs sans risque, $1-\alpha$ actifs risqués
    \end{itemize}
  }
  \only<3> {
    Le système:
  \begin{align*}
  x &= \alpha \Pi \, (1+R) +(1-\alpha) \Pi \, u \\
  y &= \alpha \Pi \, (1+R) + (1-\alpha) \Pi \, d
  \end{align*}
  a une  unique ssi $d < 1+R < u$
  $$ \alpha  = \frac{x-y}{u-d}$$
  $$\Pi = p^Q \frac{u}{1+R} + (1-p^Q) \frac{d}{1+R}$$
}
\end{frame}

\subsection{Formulation mathématique}
\begin{frame}
  \frametitle{Prix d'un actif financier}
  En général:
$$ \Pi_t = \espr{ e^{-\int_t^T r_s \rm{d}s} \, \Pi_T | F_t} $$
\end{frame}

\begin{frame}
  \frametitle{Changement de mesure}
  Changer l'unité dans laquelle les prix sont exprimés
  $$\Qespr{Q^N}{ H } = \Qespr{Q^M}{ \frac{ M_T/N_T}{M_0/N_0} H}$$
  Exemple:
  \begin{itemize}
  \item Mesure forward neutre,$P(t, T)$ comme numéraire. $\Pi_t = P(t, T) \, \Qespr{Q_T}{ H | F_t } $
  \item Mesure annuité, $A(t) := \sum\limits_{\text{Période} T_i} c_i P(t, T_i)$ comme numéraire. 
  \end{itemize}
\end{frame}

\begin{frame}
  \frametitle{Prix d'un call sur zéro coupon}
  $$ZBC(t, T, S, K) := \espr{ e^{-\int_t^T r_s \rm{d}s} \, (P(T, S) - K)^+ | F_t }$$
  Sous mesure forward neutre:
  $$ZBC(t, T, S, K) = P(t, T) \, \Qespr{Q_T}{(P(T, S) - K)^+ | F_t}$$
  Similarités avec le modèle de Black \& Scholes.
\end{frame}



\begin{frame}
  \frametitle{Produits financiers}
  \begin{itemize}
  \item Caps: Somme de caplets $$ N \sum D_t^{T_i} \tau_i (L(T_{i-1}) - K)^+ $$
  \item Swaptions: $$ N \left[ \sum_{i} D_{T_{\alpha}}^{T_i} (T_{i+1} - T_i) \left(L(T_{\alpha}, T_i) - K \right) \right]^+$$
  \end{itemize}
\end{frame}

\begin{frame}
  \frametitle{Pricing}
  \begin{align*}
    CPL(t, T, S, \tau, X)
    &= \espr{ e^{-\int_t^S r_s \rm{d}s} \tau (L(T, S) - X)^+ | F_t} \\ 
    &= \espr{ e^{-\int_T^S r_s \rm{d}s} P(t, T)  \tau (L(T, S) - X)^+ | F_t} \\
    &= \espr{ e^{-\int_T^S r_s \rm{d}s} (1 - (1 + X \tau)P(t, T))^+ | F_t} \\
    &= (1 + X \tau) \espr{ e^{-\int_T^S r_s \rm{d}s} (\frac{1}{1 + X \tau} - P(t, T))^+ | F_t} \\
    &= (1+X \tau) ZBP(t, T, S, \frac{1}{1+X \tau})
\end{align*}
\end{frame}

\subsection{Le modèle G2++}
\begin{frame}
  \frametitle{Pourquoi une simple interpolation ne marche pas}
  \begin{itemize}
  \item Estimation de risque impossible
  \item Impossibilité de faire du hedging
  \end{itemize}
\end{frame}

\begin{frame}
  \frametitle{Le modèle de Hull White à un facteur}
  Décrire la dynamique de 
$$r_t = \beta (\theta - r_t)  \mathrm{d}t  + \sigma \mathrm{d}W_t$$
$$P(t, T) = A(t, T) exp(-B(t, T) r_t)$$
\only<2> {$$R(t, T) = \frac{ln P(t, T)}{T-t} =: a(t, T) + b(t, T) r_t$$
$$Cor(R(0, T_1), R(0, T_2)) = 1$$}
\end{frame}

\begin{frame}
  \frametitle{Intérêt du modèle à deux facteurs}
\IMG{img/tabcorr.png}{Matrice de corrélation de taux à différentes maturités}{0.3}
\end{frame}

\begin{frame}
\frametitle{ODE}
\begin{align*}
  \mathrm{d}x &= -\beta^y x(t) \rm{d}t + \sigma^x \rm{d}W^x_t \\
  \mathrm{d}y &= -\beta^y y(t) \rm{d}t + \sigma^y \rm{d}W^y_t \\
  r(t) &= h(t) + x(t) + y(t)
\end{align*}
$$\rho \, \rm{d}t = < \rm{d} W^x, \rm{d} W^y >$$
\end{frame}

\begin{frame}
  \frametitle{Simulation par un arbre}
  %%% Local Variables:
%%% mode: latex
%%% TeX-master: "main.tex"
%%% End:

\tikzstyle{bag} = [text width=10em, text centered]
\tikzstyle{end} = []
\begin{figure}[H]
  \centering
\begin{tikzpicture}[sloped]
  \node (xy) at (0,0) [bag] {$(\widetilde{x}, \widetilde{y})$};
  \node (a) at ( 3,-2) [bag] {$(\widetilde{x} + \rm{d}x, \widetilde{y} + \rm{d}y)$};
  \node (b) at ( -3,-2) [bag] {$(\widetilde{x} + \rm{d}x, \widetilde{y} - \rm{d}y)$};
  \node (c) at ( 3,2) [bag]{$(\widetilde{x} - \rm{d}x, \widetilde{y} + \rm{d}y)$};
  \node (d) at ( -3,2) [bag]{$(\widetilde{x} - \rm{d}x, \widetilde{y} - \rm{d}y)$};
  \draw [->] (xy) to node [below] {$p^{+, +}$} (a);
  \draw [->] (xy) to node [below] {$p^{+, -}$} (b);
  \draw [->] (xy) to node [above] {$P^{-, +}$} (c);
  \draw [->] (xy) to node [above] {$p^{-, -}$} (d);
\end{tikzpicture}
\end{figure}




\end{frame}

\begin{frame}
  \frametitle{Table de transitions}
  $$a, b \in \{-1, 1\}$$
  $$\mathrm{P} \left( \widetilde{x}_{i+1} = \widetilde{x}_i + a \, \mathrm{d}x, \widetilde{y}_{i+1} = \widetilde{y}_i + b \, \mathrm{d}y |  \widetilde{x}_i, \widetilde{y}_i \right) = p^{a, b}( \widetilde{x}_i, \widetilde{y}_i)$$
  
$$ p^{a, b}(x, y) = \frac{1 + a \rho}{4} - b \frac{\beta^y \sigma^x y + a \sigma^x \sigma^y  x}{4 \sigma^x \sigma^y} \sqrt{\Delta t} $$
\end{frame}

\begin{frame}
  \frametitle{Lattice de simulation}
  \includegraphics<1>[scale=0.3]{img/slices2d/sl_1.png}
  \includegraphics<2>[scale=0.3]{img/slices2d/sl_10.png}
  \includegraphics<3>[scale=0.3]{img/slices2d/sl_19.png}
  \includegraphics<4>[scale=0.3]{img/slices2d/sl_46.png}
\end{frame}

\begin{frame}
  \frametitle{Limites de l'arbre}
  $$||W^x_t + W^y_t||^2 \leq n_{\sigma}t$$
  \IMG{img/slicetree.png}{Slice 3D}{0.4}
\end{frame}

\section{Calibration}
\begin{frame}
  \frametitle{Dynamique des zéro coupons}
  $$P(t, T) = \espr{ e^{-\int_t^T r_u \rm{d}u} | F_t} $$
  $$ \int_t^T x(t) + y(t) \rm{d}u | F_t \sim \mathcal{N}(M(t, T), V(t, T)) $$
  où:
  $$ M = M^x(t, T) x(t) + M^y(t, T) y(t) $$
  $$P(t, T) = e^{-\int_t^T h -M + \frac{1}{2}V} $$
\end{frame}


\begin{frame}
  \frametitle{Calibration sur la courbe des zéro coupons}
  \begin{align}
    \Pi_j(t_{n+1}) &= \sum_{j' \text{noeud}} \Pi_{j'}(t_n) p_{j', j}(t_n) D_{j'}(h_n) \\
    \sum_{ j \text{noeud} } \Pi_j D_j(h_n) &= P^M(0, t_{n+1})
  \end{align}
\end{frame}


\begin{frame}
 \frametitle{Prix de caplets}
 \begin{align*}
   ZBC(t, T, \tau, K) &= \Qespr{Q_T}{ (P(t, S)-K)^+ | F_t} \\
                      &= -P(t, T) N( d_1 ) + P(t, T) K N(d_2)
\end{align*}
où
$$d_{1/2} := \frac{ln \frac{KP(t, T)}{P(t, S)}}{\Sigma} +/- \frac{1}{2}\Sigma $$
$$\Sigma^2 := \Sigma^{x,x} + \Sigma^{y,y} + 2 \rho \Sigma^{x,y}$$
$$\Sigma^{x,y} := \sigma \nu M^x(t, T) M^y(t, T) \frac{1 - e^{(\alpha+\beta) (T-t)}}{\alpha+\beta} $$

\end{frame}

\begin{frame}
  \frametitle{Prix de swaption}
  \begin{adjustwidth}{-2em}{-2em}
    On définit le taux swap :
    $$\sum\limits_{\text{Période} T_i} \tau_i P(t, T_i) S_{\alpha, \beta} (t) = P(t, T_\alpha) - P(t, T_\beta)$$
    $S_{\alpha, \beta}(t)  A(t)$ est donc une martingale.
    \only<2> {
      \begin{align*}
        Swaption(t; \alpha,\beta)
        &= \espr{ \left( \sum_{i} D_t^{T_i} \tau_i \left(L(T_{\alpha}, T_i) - K \right) \right)^+ | F_t } \\
        &= A(0) \Qespr{Q_a}{ (S_{\alpha, \beta}(t) - K)^+ | F_t}
      \end{align*}
    }
  \end{adjustwidth}
\end{frame}

\begin{frame}
  Benchmark
  \begin{table}
    \centering
\begin{tabular}{|l|c|r|}

  \hline
  Index&Maturity&Price \\
  \hline
  0&1&0.972411 \\
  1&2&-1.334463 \\
  2&3&0.614967 \\
  3&4&0.240523 \\
  4&5&0.877871 \\
  5&7&-0.665534 \\
  6&10&-0.151002 \\
  7&15&0.050241 \\
  8&20&0.133609 \\
  \hline
\end{tabular}
\caption{Prix marché de caps}
\end{table}
%%% Local Variables:
%%% mode: latex
%%% TeX-master: "main"
%%% End:

\end{frame}

\begin{frame}
  \frametitle{Calibration - Caps}
  \begin{columns}
    \begin{column}{0.4\textwidth}
      \IMG{img/calibrationcap.png}{Calibration des caps}{0.2}
    \end{column}
    \begin{column}{0.4\textwidth}
      \centering
      \begin{tabular}{ l |c| c }
        paramètre & x & y  \\
        \hline \hline
        $\beta$ & 0.62 & 0.025 \\
        $\sigma$ & 0.69\% & 0.81\% \\
        $\rho$ & 96\% & - \\
      \end{tabular}
      \only<2>{$$ \text{erreur} = 0.98\%$$}
    \end{column}
​  \end{columns}
\end{frame}

\begin{frame}
\begin{table}[H]
    \centering
\begin{tabular}{|c|c|c|c|c|c|c|c|c|c|r|}
  \hline
  &3m&1y&2y&3y&4y&5y&7y&10y&15y&20y\\
  \hline

3m& 7.55 & 12.94 & 15.44 & 16.40 & 15.67 & 16.23 & 14.94 & 14.46 & 12.85 & 13.76\\
6m & 11.63 & 15.80 & 16.59 & 15.82 & 16.33 & 16.06 & 15.54 & 14.82 & 13.83 & 12.88\\
1y& 18.27 & 16.71 & 17.39 & 16.88 & 16.32 & 16.83 & 16.12 & 15.70 & 14.66 & 13.94\\
2y&18.42 & 18.20 & 17.58 & 17.49 & 17.84 & 17.39 & 15.96 & 15.38 & 15.07 & 14.35\\
3y&19.44 & 18.38 & 18.30 & 18.50 & 17.84 & 16.98 & 17.35 & 16.43 & 15.18 & 14.56\\
4y&19.27 & 18.51 & 17.37 & 18.35 & 18.27 & 16.92 & 16.80 & 16.41 & 15.23 & 14.30\\
5y&18.41 & 18.29 & 18.35 & 17.01 & 17.35 & 17.39 & 17.48 & 15.34 & 14.18 & 14.17\\
7y&17.59 & 17.05 & 17.47 & 17.37 & 16.94 & 16.47 & 15.41 & 15.68 & 13.74 & 14.26\\
10y&16.70 & 15.94 & 16.30 & 14.69 & 15.65 & 15.47 & 14.50 & 14.38 & 12.74 & 12.32\\
  \hline
\end{tabular}
\caption{At the money Euro swaption-volatillity quotes on February 13, 2001, at 5p.m.}
\end{table}
  
%%% Local Variables:
%%% mode: latex
%%% TeX-master: "main"
%%% End:

\end{frame}

\begin{frame}
  \frametitle{Calibration - Swaption}
    \begin{columns}
      \begin{column}{0.4\textwidth}
        \IMG{img/swapcalib.png}{Prix de marché en bleu, prix du modèle en rouge }{0.2}
    \end{column}
    \begin{column}{0.4\textwidth}
      \centering
      $$ \rho = -71\%$$
      \only<2>{$$ \text{erreur} = 1.12\%$$}
    \end{column}
​  \end{columns}

\end{frame}


\section{Méthodes d'optimisation avancées}
\begin{frame}
  \frametitle{Paramètres dépendant du temps}
  Espace de grande dimension:
  $$ \sigma = (\sigma_{t_1}, ..., \sigma_{t_n}) $$
  $$ \beta = (\beta_{t_1}, ..., \beta_{t_n}) $$
  $$ \rho = (\rho_{t_1}, ..., \rho_{t_n}) $$

  Calibration = résoudre
  $$F(\sigma, \beta, \rho) = \sum_{\text{produit} \, i} ||P_i(\sigma, \beta, \rho) - P_i^{\text{Market}} ||^2$$
  Ou encore:
  $$G(x) = 0$$
\end{frame}

\begin{frame}
  \frametitle{Hypothèses}
  Quelques remarques sur $G$:
  \begin{itemize}
  \item $G$ définie sur un compact convexe $\Omega$, a dérivée continue
  \item Les appels à $G$ sont coûteux (construction de l'arbre, calcul de probabilités ...)
  \end{itemize}
\end{frame}

\begin{frame}
  \frametitle{Descente du gradient}
  \begin{itemize}
  \item Calcul de $\nabla G(x_k)$
  \item Résoudre le problème de minimisation à une seule dimension
    $\alpha_k = \text{argmin} ||G\left(x_k - \alpha_k \nabla G (x_k) \right)||$
  \item $ x_{k+1} = x_k - \alpha_k \nabla G(X_k)$
  \end{itemize}
Condition d'\textbf{arrêt}: $||\nabla G(x_k)|| <  \epsilon$
\end{frame}

\begin{frame}
  \frametitle{Algorithmes génétiques}
  \begin{columns}
    \begin{column}{0.4\textwidth}
      \begin{itemize}
      \item Individus: $\theta = ( \sigma^x_t, \sigma^y_t, \beta^x_t, \beta^y_t, \rho_t)_{t = T_\alpha \dots T_\beta}$
      \item Crossover
      \item Mutation
      \end{itemize}
    \end{column}
    \begin{column}{0.5\textwidth}
      \IMG{img/gaalgo.png}{Algorithme génétique}{0.3}
    \end{column}
    \end{columns}
\end{frame}


\begin{frame}
  \frametitle{Résultats empiriques}
  \only<1>{
  One max problem
  \begin{displaymath}
    f: \left.
  \begin{array}{rcl}
    \{0, 1\}^n & \longrightarrow &\mathbf{N} \\
    x & \longmapsto & \sum\limits_{i=1 \dots n} x_i \\
  \end{array}
\right.
\end{displaymath}
}
\only<2>{\IMG{img/onemax.png}{Résultats empiriques}{0.4}}
\end{frame}

\section*{Conclusion}
\begin{frame}
  
  \begin{itemize}
  \item Conclusion
  \item<2>{Questions}
  \end{itemize}
\end{frame}

\end{document}

%%% Local Variables:
%%% mode: latex
%%% TeX-master: t
%%% End:
