%%% Preamble
\documentclass[paper=a4, fontsize=11pt]{scrartcl}
\usepackage[utf8]{inputenc}  
\usepackage[T1]{fontenc}
\usepackage{fourier}
\usepackage[english]{babel}															% English language/hyphenation
\usepackage[protrusion=true,expansion=true]{microtype}	
\usepackage{amsmath,amsfonts,amsthm} % Math packages
\usepackage[pdftex]{graphicx}	
\usepackage{url}
\usepackage{tikz}

%%% Custom sectioning
\usepackage{sectsty}
\allsectionsfont{\centering \normalfont\scshape}

%%% Custom headers/footers (fancyhdr package)
\usepackage{fancyhdr}
\pagestyle{fancyplain}
\fancyhead{}											% No page header
\fancyfoot[L]{}											% Empty 
\fancyfoot[C]{}											% Empty
\fancyfoot[R]{\thepage}									% Pagenumbering
\renewcommand{\headrulewidth}{0pt}			% Remove header underlines
\renewcommand{\footrulewidth}{0pt}				% Remove footer underlines
\setlength{\headheight}{13.6pt}


%%% Equation and float numbering
\numberwithin{equation}{section}		% Equationnumbering: section.eq#
\numberwithin{figure}{section}			% Figurenumbering: section.fig#
\numberwithin{table}{section}				% Tablenumbering: section.tab#


%%% Maketitle metadata
\newcommand{\horrule}[1]{\rule{\linewidth}{#1}} 	% Horizontal rule
\newcommand{\espr}[1]{
  \mathrm{E}^Q \left[ #1 \right]
}


\title{
		%\vspace{-1in} 	
		\usefont{OT1}{bch}{b}{n}
		\normalfont \normalsize \textsc{Ecole Polytechnique} \\ [25pt]
		\horrule{0.5pt} \\[0.4cm]
		\huge Modèles des Taux d'interêt \\
		\horrule{2pt} \\[0.5cm]
}
\author{
		\normalfont \normalsize
                Bachir EL KHADIR\\[-3pt] \normalsize
                \today	
}
\date{}

\theoremstyle{definition}
\newtheorem{theorem}{Theorem}
\newtheorem{defn}{Definition}

\newcommand{\IMG}[3]{
  \begin{center}
    \includegraphics[scale=#3]{#1}%
    \end{center}
}


%%% Begin document
\begin{document}
\maketitle

\newpage
\tableofcontents

\newpage

%%% Local Variables:
%%% mode: latex
%%% TeX-master: "main.tex"
%%% End:

\chapter*{Remerciments}

Je tiens à remercier toutes les personnes qui ont contribué au succès de mon stage et qui m'ont aidé lors de la rédaction de ce rapport.

Tout d'abord, j'adresse mes remerciements à mon professeur, Mr Stefano De Marco de l'Ecole Polytechnique qui m'a beaucoup aidé dans ma recherche de stage et ma permis de postuler dans cette entreprise. Son écoute et ses conseils m'ont permis de cibler mes candidatures et de trouver ce stage me correspondait totalement.

Je tiens à remercier vivement mon maitre de stage, Mr John Paul Barjaktarevic, responsable du service Y au sein de la banque JP Morgan, pour son accueil, le temps passé ensemble et le partage de son expertise au quotidien. Grâce aussi à sa confiance j'ai pu m'accomplir totalement dans mes missions avec son aide précieuse dans les moments les plus délicats.

Je remercie également toute l'équipe ``rates exotic'' pour leur accueil, leur esprit d'équipe. 

Enfin, je tiens à remercier toutes les personnes qui m'ont conseillé et relu lors de la rédaction de ce rapport de stage : ma famille, mon amie Julie B camarade de promotion.


\newpage

%%% Local Variables:
%%% mode: latex
%%% TeX-master: "main"
%%% End:

\newpage

\subsection*{Résumé}
  
  Ce rapport présente quelques outils utilisés lors du pricing de dérivés de taux. 
  Il est particulièrement centré sur le modèle de Hull White qui est conduit par deux facteurs (appelé aussi G2++).  J'adopterai le plan suivant dans la rédaction de ce rapport:
  \begin{itemize}
  \item 
    La première partie introduit le marché de taux par un certain nombre de définitions qui lui sont propores, ainsi que les produits financiers principaux utilisés pour la calibration des modèles.
  \item 
    Dans la deuxième partie je présenterai quelques modèles existant, et je discuterai de leurs limites et de leurs implémentation pratique.
  
  \item
    La troisième partie présente les résultats concrets que j'ai obtenus dans le cas du modèle gaussien à deux facteurs, ainsi que la partie calibratoin de modèle.
  \end{itemize}
  
\subsection*{Abstract} 
This report presents some tools used in the pricing for intereset rates derivatives. It's focused on two factors Hull-White model (called G2++).  
  \begin{itemize}
  \item
    The first part introduces some definitions relative to the interest rates market, as well as some derivative used later for the calibration process.
  \item
    In the second part I will present some existing models, their limitations and possible implementation.

  \item
    The last part presents the results I obtained using the G2++ model and the calibration process.
  \end{itemize}




\newpage
\chapter*{Introduction}

J'ai fait mon stage a JP Morgan equipe exotic rates. La plus grosse partie du travail consitait à trouver le bon modèle mathématique pour pricer un produit et le calibrer au marche.

L'interêt que porte les banques aux produits exotiques a considérablement augmenté aucours des dernières années. Plusieurs modèles ont été dévelopé pour refeléter au mieux le comportement des marchés financiers.

Le département au sein duquel j'ai effectué mon stage s'occupait des produits basé sur les taux d'intêret. Dans ce domaine, une propriété appréciable dans un modèle est le fait qu'il possèd un nombre de paramètres suffisant pour être calibré parfaitement aux prix observable dans la réalité, sans pour autant ``overfitter'' l'échantillon disponible. En effet, dérnièrement, surtout depuis la crise, le marché des taux a connu des changement radicaux, la possiblité d'observer des taux négatifs en est un example. 

Un produit en particulier n'a cessé de  gagner en notoriété: les cancelables spread options. Ma mission de stage était de comprendre les modèles existant et leurs implémentations qui permettent de pricer (entre autres) ce type de produits, comprendre leurs limites, et essayer de trouver des améliorations.

Ce stage est à forte composante informatique, une attention particulière a été accordée aux détails d'implémentation et optimisation du code.
 
\newpage
%%% Local Variables:
%%% mode: latex
%%% TeX-master: "main"
%%% End:


\chapter{Préliminaires sur les taux d'intérêt}

\section{Définition}
\begin{defn}
On dénote le prix de zéro d'une obligation zéro-coupon $P(T, S)$ le montant qu'il faut investir à dans un instrument risque-neutre au temps $T$ pour obtenir une unité de monnaie au temps $S$.
\end{defn}

\begin{defn}
On définit $f(t, T)$ le taux d'intérêt instantané forwad à la date $t$ pour une maturité $T$ la quantité $$f(t, T) := - \frac{ \delta}{\delta T}  log P(T, S)$$
\end{defn}

\begin{defn} Le taux instantané est défini par
  $r_t = \underset{T \to t}{lim}f(t, T) $ \\
  Le taux d'actualisation (stochastique) est: $D_t^T := e^{-\int_t^Tr_s \rm{d}s}$
\end{defn}

\iffalse
\begin{defn}
  Le taux d'intérêt cumulé entre deux période $t$ et $T$ est la quantité $R(t, T)$ que $r_t$ doit égaler pour avoir le même rendement
\end{defn}
\fi

\begin{defn}
  Le taux d'intérêt forward $F(t; T, S)$ est la prévision  à l'instant $t$ du taux entre deux période $T$ et $S$.
  Ce taux est la quantité $L$ connu à l'instant $t$ qui annule la valeur du contrat suivant à l'instant $t$:
  \begin{itemize}
  \item Recevoir l'intérêt  $L$ sur un 1 euro entre $T$ et $S$
  \item Payer le taux variable  $F(T, S)$ sur un 1 euro entre $T$ et $S$
  \end{itemize}
\end{defn}


le taux $r_t$ n’est pas un produit échangé sur le marché que l’on peut mettre en portefeuille. On ne peut donc pas construire de couverture d’un produit donné de la même manière que dans un modèle d’action, et ce malgré la similitude des modèles mathématiques.

\section{Mesures équivalentes}

Nous nous plaçons dans le cadre d'une économie à temps continu, qui admet un espace de probabilité $(\Omega, \cal{F}, \mathbb{P})$, avec $K+1$ actifs tradables, que nous appellerons actifs de base, dont le prix est donné par $(S_t = (S^0_t, ...S^k_t))_{t \geq 0}$. Dans toute la suite nous confondons l'actif et son prix.
$S^0$ étant l'actif sans risque, qui évolue donc au temps sans risque $$\mathrm{d}S^0_t = r_t S^0_t \mathrm{d}t$$
ie $$S^0_t = e^{\int_0^t r_s \mathrm{d}s}$$

Par définition, nous connaissons le prix des actifs $K+1$, dans la prochaine section nous détaillerons la procédure de pricing de produits plus compliqués.

\subsubsection{Principe de pricing}
A travers les $K+1$ actifs de base, nous construisons des produits plus complexes. 
Le prix d'un tel produit est donc intimement lié à la possibilité de trouver une stratégie autofinancée qui le réplique.
Commençons d'abord par définir ce qu'est une stratégie auto financé.

\begin{defn}
  \begin{itemize}
  \item Une stratégie est un processus $(\Phi_t = (\Phi^0_t, ... \Phi^K_t))_t$ localement borné et adapté à la filtration $\cal{F}$.
  \item La valeur associé à cette stratégie est donné par $V_t(\Phi) = <\Phi_t, S_t>$.
  \item Une stratégie est auto financée sir $\mathrm{d}V_t = \Phi_t \mathrm{d}S_t$
  \end{itemize}
\end{defn}

Une hypothèse souvent utilisée dans le cadre de la finance de marché est l'absence d'arbitrage. Une opportunité d'arbitrage est la possibilité d'investir 0 aujourd'hui, et recevoir, avec probabilité non nulle, un montant positive dans le future. En d'autres termes, l'absence d'arbitrage signifie que si $\Phi$ est une stratégie auto financée telle que $V_0(\Phi) = 0$, alors $\mathbb{P} ( V_t(\Phi) > 0 ) = 0$. Ceci nous permettra de valoriser des produits complexes en répliquant leur payoff par une combinaison linéaire de produits simples dont le prix est connue.

Une deuxième hypothèse que nous admettrons dans la suite est la complétude du marché: Tout les produits utilisés seront considérés disponibles à tout moment et en quantité abondante (liquide), ie à chaque instant $t$, pour tout payoff $H$, il existe une stratégie autofinancée associée $\Phi$ qui vérifie $V_t( \Phi ) = H$. Nous ne traiterons pas le cas des produits illiquides. Ceci est justifié, le marché des taux étant l'un des plus gros en volume dans le monde.

Nous pouvons montrer( \cite{Brugo}) que ces hypothèse sont équivalentes à l'existence d'une mesure de probabilité risque neutre $Q$ unique sous laquelle le prix actualisé de tous les produits tradables sont des martingales. ie si un on note $H_t$ le prix à l'instant $t$ d'un produit financier, alors
$$H_t =\espr{ \frac{H_s}{S^0_t} | F_t } = V_t(\Phi)$$

En particulier, le prix d'un zéro coupon qui paye 1 à l'instant $T$ est donné par
$$ P(t, T) := \espr{  e^{-\int_t^T r} } $$


Nous pouvons interpréter le ratio $\frac{H_s}{ S^0_s}$ comme étant le nombre  de $H$ par unité de facteur d'actualisation stochastique $S^0$. Le facteur d'actualisation est appelé dans ce cas numéraire. Nous verrons maintenant que nous pouvons choisir un autre numéraire plus adapté au produit que nous voulons pricer, puisque le changement de numéraire préserve la propriété d'autofinancement d'un portefeuille. \cite{Hull}

\begin{defn} Un numéraire est tout actif financier ne payant pas de dividendes \end{defn}

\begin{defn} Mesure de probabilité équivalente.
  
Supposons qu’il existe un numéraire $(M_t )_{t \geq 0}$ et une mesure martingale équivalente $Q^M$ telle que le prix de chaque actif actualisé par le processus M soit une$Q_M$-martingale. 
$$  (\forall i) \frac{S^i_t}{M_t} = \Qespr{Q^M}{ \frac{S^i_T}{M_T} | F_t}$$
Soit $(N_t )_{t \geq 0}$ un numéraire. \\
Alors il existe une mesure de probabilité $Q_N$ telle que le prix de chaque actif actualisé par le processus $N$ soit une $Q_N$-martingale, ie.
$$ (\forall i) \frac{S^i_t}{N_t} = (\forall i) \Qespr{Q^N}{ \frac{S^i_T}{N_T} | F_t}$$

où $Q^N$ est définie par:
$$\Qespr{Q^N}{ H } = \Qespr{Q^M}{ \frac{ M_T/N_T}{M_0/N_0} H}$$

\end{defn}

\textbf{Exemple:} Mesure forward neutre

Le bond zéro coupon dont la maturité coïncide avec la date du payement d'un produit financier peut servir de numéraire. Nous appellerons la mesure de probabilité associé $Q_T$.

Dans ce cas $P(T, T) = 1$, et par conséquent il suffit de calculer l'espérance du payoff (divisé par 1) sous $Q_T$.
Si nous notons le payoff de ce produit $H$, alors son prix à l'instant $0$ est donné par $$P(t, T) \, \Qespr{Q_T}{ H | F_t } $$
Pour que cela nous soit utile, il faut que la dynamique de $H$ soit connue sous $Q_T$. Ceci est vérifié pour les contrats payant un taux d'intérêts sur un nominal fixe. En effet $(F(t; S, T))_t$ est une martingale 
$$ \Qespr{Q_T}{ F(t; S, T) | F_u } = F(u; S, T)$$

\textbf{Preuve:}
Si nous disposons de  $\frac{P(t, S)}{1+(T-S)F(t; S, T)}$ au temps $t$, nous pouvons acheter $\frac{1}{1+(T-S)F(t; S, T)}$ unités de l'obligation $P(t, S)$, nous obtenons $\frac{1}{1+(T-S)F(t; S, T)}$ au temps $S$, cette somme là est, par définition de $F$, équivalente à l'obtention de 1 à l'instant $T$, qui exactement le payoff de l'obligation $P(t, T)$.
Par principe de \textbf{non arbitrage}, ces deux investissement doivent avoir le même coût, ie: $$\frac{P(t, S)}{1+(T-S)F(t;S,T)} = P(t, T)$$, ou encore
$$ \frac{1}{T-S} \left( \frac{P(t, S)}{P(t, T)} - 1  \right) $$
La preuve en découle.

\newpage

\section{Produits financier d'intérêt}

Le développement de la section précédente nous sera utile pour pricer les dériver des taux.
Considérons le cas particulier d'un call européen à maturité $T$, strike $K$, dont le sous-jacent est bond zéro coupon qui expire à l'instant $S$. Le payoff d'un tel contrat est connu: $ (P(T, S) - K)^+)$. Son prix à un instant antérieur $t$ est
$$ZBC(t, T, S, K) := \espr{ e^{-\int_t^T r_s \rm{d}s} \, (P(T, S) - K)^+ | F_t }$$
Il est plus pratique de considérer la forward mesure, sous laquelle le prix du call s'écrit
$$ZBC(t, T, S, K) = P(t, T) \, \Qespr{Q_T}{(P(T, S) - K)^+ | F_t}$$
De même, pour un put
$$ZBP(t, T, S, K) = P(t, T) \, \Qespr{Q_T}{(K - P(T, S))^+ | F_t}$$

Cette écriture nous rappelle la formule de blackscholes pour les options sur les actions.

\begin{defn}
  Swap:
Un swap est un contrat entre deux parties qui s'engagent à échanger des flux financiers pendant une durée et à une fréquence déterminées. La plupart du temps, ces flux sont déterminé comme étant l'intérêt à un taux fixe $K$ contre un taux variable (taux Libor ${L(T_i)}_i$ par exemple) sur un notionnel $N$. 
$$ N \sum D_t^{T_i} \tau_i (L(T_{i-1}) - K) $$
\end{defn}


\begin{defn}
  Caplet:
  Un caplet peut être vu comme un call/put européenne sur un
  Son payoff est le suivant
$$ \tau (L(T, S) - K)^+ $$
\end{defn}

\begin{align}
  Cpl(t, T, S, \tau, X)
  &= \espr{ e^{-\int_t^S r_s \rm{d}s} \tau (L(T, S) - K)^+ | F_t} \\ 
  &= \espr{ e^{-\int_T^S r_s \rm{d}s} P(t, T)  \tau (L(T, S) - K)^+ | F_t} \\
  &= \espr{ e^{-\int_T^S r_s \rm{d}s} (1 - (1 + X \tau)P(t, T))^+ | F_t} \\
  &= (1 + X \tau) \espr{ e^{-\int_T^S r_s \rm{d}s} (\frac{1}{1 + X \tau} - P(t, T))^+ | F_t} \\
  &= (1+X \tau) ZBP(t, T, S, \frac{1}{1+X \tau})
\end{align}

\begin{defn}
  Cap:
Un cap peut être vu comme une somme de caplets
$$ N \sum D_t^{T_i} \tau_i (L(T_{i-1}) - K)^+ $$
\end{defn}

La forme des payoff indique que le cap permet de protéger son détenteur d’une hausse des taux Libor, et symétriquement que le floor protège d’une éventuelle baisse de ces taux.

\begin{defn}
  Swaption:

 Une swaption payeuse européenne est une option permettant d’entrer, à une date $T$ appelé maturité, dans un swap payeur pour la période $(\alpha, \beta)$ de nominal $N$ et de strike $K$. 
 Les pâment s'effectuent aux instants $T_{\alpha} \leq T_i \leq T_{\beta}$. A la date $T_\alpha$, la valeur du swap payeur sous-jacent s'écrit:
$$ N \sum_{i} P(T_{\alpha}, T_i) (T_{i+1} - T_i) \left(L(T_{\alpha}, T_i) - K \right)$$
La valeur de la swaption s'écrit alors:
$$ N \left[ \sum_{i} P(T_{\alpha}, T_i) (T_{i+1} - T_i) \left(L(T_{\alpha}, T_i) - K \right) \right]^+$$
  
\end{defn}

Il est intéressant de noter que l'expression d'un swaption en fonction des taux sous-jacents n'est plus linéaire comme dans le cas des caps. 

%%% Local Variables:
%%% mode: latex
%%% TeX-master: "main"
%%% End:
 
\newpage
\section{Les différents modèles des taux d'interêts}

Nous avons vu dans la partie précédente que pour calculer l'espérance, il faut donner la dynamique du produit sous-jacent.
Plusieurs choix sont disponilbes:
\begin{itemize}
\item Long term interest rate
\item Short term
\end{itemize}

Certains produits financiers dépendent directement la courbe de rendement. Nous avons vu précédemment que la donnée du taux instantané $r_t$ permet de caractériser complètement cette courbe. 

Il est donc important que la dynamque de $r_t$ soit à la fois riche pour pouvoir décrire la courbe de rendement observée dans le marché, et suffisemment simple pour que le temps nécessaire pour le calcul ne soit trop long.

Nous pouvons considérer que la courbe de rendement varie dans un espace vectoriel de dimension infinie. Toute tentative de la caracteriser par un nombre fini de paramètre rééls est donc vouée à l'échec.

Le modèle de Hull White a été introduit en 1990.  Un des atouts majeure de ce modèle est la possiblité de simuler la dynamique de $r_t$ par un arbre trinomiale. Ceci étant essentiel pour pricer des produits du type bermuda options
$$ \mathrm{d}r_t =  (\theta(t) - \alpha(t) r_t) \mathrm{d}t + \sigma(t) \mathrm{d} W_t$$
$$ r_t = e^{-\alpha t} r_0 + integral ...$$


\newpage
\section{Le modèle à deux facteurs}
\subsection{Motivation}


Considérons un produit $E$ dont le payoff  dépent  du spread entre un taux d'intêret cumulé entre $0$ et $T_1$ pour le premier et $0$ et $T_2$ pour le second. $E$ dépend donc de la distribution jointe des deux taux.

\subsubsection*{Limite des modèle à un seul facteur}
Montrons dans un premier temps pourquoi un modèle à un seul facteur n'est pas suffisant pour pricer ces produits qui dépendent non seulement de la distribution de chaque courbe de taux, mais aussi de leur corrélation. 

La dynamique de $r_t$ dans le modèle de Vasicek est donné par
$$r_t = k(\theta - r_t)  \mathrm{d}t  + \sigma \mathrm{d}W_t$$
La formule analytique du bond zéro coupon est donc
$$P(t, T) = A(t, T) exp(-B(t, T) r_t)$$
En particulier le taux d'intêret cumulé est une donné par une transformation affine du taux instantané:
$$R(t, T) = \frac{ln P(t, T)}{T-t} =: a(t, T) + b(t, T) r_t$$
Le payoff du produit $E$ est donc fonction de la distribution jointe de $R(0, T_1)$ et $R(0, T_2)$. Sauf que:
$$Cor(R(0, T_1), R(0, T_2) = 1$$

On en déduit qu'un choc à $r_t$ agit de la même manière sur toutes les courbes.

Dans la réalité, on observe sur les marchés que les taux à différentes maturités ne sont pas corrélés. Si on regarde le taux 2Y et 10Y

\IMG{img/libor.png}{Libor}{1}

Un modèle à un seul facteur ne capture pas ce comportement. Essayons de pallier à ce problème en rajoutons un facteur à ce modèle.

Dans cette section nous considérons un modèle où le taux d'intêret instantanté est donné par une somme de deux facteurs gaussiens centrés et corrélés. 
\begin{align*}
  \rm{d}X &= \\
  \rm{d}Y &= \\
  \rm{d} r &= X + Y + \phi(t)
\end{align*}

Pour fitter la courbe on rajoute une fonction détérministe.


$$P(t, T) = E[...]$$

Le processus est markovien

Les cap floors ademettent une formule analytiques, comme pour les options vanilles : formule de blackscholes
Dans la pratique, les banques tradent beaucoup des produits exotiques. Ce n'est pas le cas pour ces produits => arbre de pricing

\newpage
\subsection{Approximation de la solution par un arbre binomial}

Cette méthode a été d'abord suggéré par Hull-White (1994)
On commence par donner une approximation de la dynamique processus $x$ et $y$, par une suite de variable de discretisé $((\widetilde{x}_i, \widetilde{y}_i) \approx (x(i \Delta i), (y(i \Delta t))_i $. Pour celà nous calculerons les deux premier moment de $(x, y)$

$$E(x(t+\Delta t) | F_t) = x(t) e^{-a \Delta t}$$
$$V(x(t+\Delta t) | F_t) = \frac{\sigma^2}{2a} (1 - e^{-2a \Delta t})$$
$$Cov\{x(t+\Delta t), y(t+\Delta t) | F_t \} = \frac{\sigma \nu \rho}{a + b} (1-e^{-(a+b)\Delta t})$$

Pour que le $((\widetilde{x}_i, \widetilde{y}_i)$ et $(x(i \Delta i), (y(i \Delta t))_i $ aient les même moment, la loi de  $((\widetilde{x}_i, \widetilde{y}_i)$  est donnée par:
$$\mathrm{P} \left( \widetilde{x}_{i+1} = \widetilde{x}_i + a \, \mathrm{d}x, \widetilde{y}_{i+1} = \widetilde{y}_i + b \, \mathrm{d}y |  \widetilde{x}_i, \widetilde{y}_i \right) = p^{a, b}( \widetilde{x}_i, \widetilde{y}_i)$$
où 
\begin{itemize}
\item $a, b \in \{-1, +1\}$
\item $p$ est donnée par
\end{itemize}

Le pricing se fait en deux temps:
\begin{itemize}
\item On diffuse le processus $(\widetilde{x}, \widetilde{y})$ dans l'arbre en prenant soin de calculer la probabilité de transition d'un état à un autre
\item On ``drawback'' dans l'arbre en partant de la date à laquelle on fait le payoff, et on discount
  \end{itemize}


Si nous implémentons l'abre de façon naîve, le nombre de noeuds augmente de façon exponentielle en fonction du nombre de pas de temps. En pratique ceci est problématique et conduit vite à une saturation de mémoire. Dans l'exemple simplifié ci-dessus nous traçons l'arbre de diffusion du premier facteur ($x(t)$). A chaque pas de temps le nombre de noeuds double, ie pour $n$ pas de temps, nous nous retrouvons avec $2^n$ noeuds pour un facteur, ou $4^n$ pour deux. 

Remarquer que $(x, y)_i$tilde est un processus markovien homogène à valeurs discrètes nous permet d'optimiser la simulation de l'arbre. En effet, il nous suffit de calculer la table de transition une fois au début du programme et de la réutiliser pour avancer/reculer dans le temps. 

%%% Local Variables:
%%% mode: latex
%%% TeX-master: t
%%% End:

% Define styles for bags and leafs
\tikzstyle{bag} = [text width=2em, text centered]
\tikzstyle{end} = []
\begin{figure}[H]
  \centering
\begin{tikzpicture}[sloped]
  \node (a) at ( 0,0) [bag] {$0$};
  \node (b) at ( 4,-1.5) [bag] {$- \sigma \Delta t$};
  \node (c) at ( 4,1.5) [bag] {$+ \sigma \Delta t$};
  \node (d) at ( 8,-3) [bag] {$-2 \sigma \Delta t$};
  \node (e1) at ( 8,0.5) [bag] {$+ 0 \sigma \Delta t$};
  \node (e2) at ( 8,-0.5) [bag] {$+ 0 \sigma \Delta t$};
  \node (f) at ( 8,3) [bag] {$+ 2 \sigma \Delta t$};
  
  \draw [->] (a) to node [below] {$p^-$} (b);
  \draw [->] (a) to node [above] {$p^+$} (c);
  \draw [->] (c) to node [below] {$p^+$} (f);
  \draw [->] (c) to node [above] {$p^-$} (e1);
  \draw [->] (b) to node [below] {$p^+$} (e2);
  \draw [->] (b) to node [above] {$p^-$} (d);
\end{tikzpicture}
\caption{Arbre construit de façon naïve}
\label{oldtree}
\end{figure}




Une autre améliroation possible est de trouver une formule analytique pour certain produits. En effet, si nous reprenons l'exemple d'un cancellable spread option 2Y10Y dont la maturité est dans 5 ans, nous devrions normalement construite l'arbre jusqu'en 2030 pour avoir le taux 10Y en 2020. Nous pouvons éviter celà en fournissant directement une formule exacte pour les zéro coupons.

$$ \int r \sim Normal$$
$$P(t, T) = \espr{ e^{\int r}} = exp  \left( \espr{ \int r} + \frac{ Var [ \int r ]  }{2} \right) $$

\subsubsection{ Paramètres constants}
<Calcul >
\subsubsection{ Paramètres dépendant du temps}
<Calcul>


On vérifie expérimentalement que le gain est significatif
<Cache grind > 

\subsubsection{Code python}


\IMG{img/slice.png}{Slice}{0.5}

Cependant, on ne peut ajouter un drift au noeud avec cette methode.

Paragraphe sur la taille des slice, ellipsoid
<example de Slice>


\begin{itemize}
\item Discrétisation
\item Construction de l'abre
\end{itemize}

\subsection{Formule analytique - Intégration}

Closed form des zero coupons

en pratique, gain en temps \\
< cachegrind results > \\

gain en precision \\
<Comparaison de slice entre cf et bf>


Amerlioration:
\begin{itemize}
\item Time dependant parameters
\end{itemize}

\subsection{Performance}
L'arbre est long mais beaucoup plus puissant
Imperfections de l'arbre:
\begin{itemize}
\item  bornee
\item  discretisation
\item  probabilite negative
\end{itemize}
Monte Carlo
Limitation
Closed Form

Le temps d'execution:
Arbre
Closed form

\newpage
\section{Application: calibration et pricing}
Le modèle à 2F permet de caputrer le hump de la courbe de rendement
Le nombre de paramètre est fini (5) => pas de overfitting


\subsection{Calibration}
Dans cette section on verra a quel point le modele peut fitter la realite.

\begin{itemize}
\item $h(t)$ pour reconstruire la yield curve $\sigma \rho \nu$ pour
\item  matcher la surface
\end{itemize}

\subsubsection{Calibration du drift}

$\Pi_j$ state price (arrow debrew) (paye 1 si le noeud $(t_n, j)$ est atteint.

$$D_j(h_n) := \frac{1}{1 + r(t_n)(t_{n+1} - t_n)} $$ Le discount factor
$$ \Pi_j(t_{n+1}) = \sum_j \Pi_j(t_n) p_{j, j'}(t_n) D_{j'}(h_n)$$
$$ \sum \Pi_j D_j(h_n) = P(0, t_{n+1}) $$

Developement de taylor => trouver $h_n$


\subsubsection{Methode de calibration de la surface - Méthode d'optimisation}

On calibre les caplets

Les caplets ne sont pas directement disponilbe sur les marche
On calibre les swaptions/caps

\begin{itemize}
\item Le probleme de calbration est un probleme d'optimisation Courbe
  de vol implicite :
\item minimisation de l'erreur L2
\end{itemize}
\IMG{img/capsurf.png}{Cap surface}{0.5}

\end{document}

%%% Local Variables:
%%% mode: latex
%%% TeX-master: t
%%% End:
