\documentclass{article}

\usepackage[utf8]{inputenc} 
\usepackage[T1]{fontenc}   
\usepackage{amsmath}
\usepackage{amsfonts}
\usepackage{pgfplots}
\usepackage{sagetex}
\usepackage{graphicx}
\usepackage{caption}
\usepackage{subcaption}

\usepackage{minted}


\usepackage{geometry}
\geometry{hmargin=1.5cm,vmargin=1.5cm}

\newcommand{\BigO}[1]{\ensuremath{\operatorname{O}\left(#1\right)}}
\newcommand{\SmallO}[1]{\ensuremath{\operatorname{o}\left(#1\right)}}
\newcommand{\BBigO}[3]{\ensuremath{\underset{#1 \to #2 }{\operatorname{O}\left(#3\right)}}}
\newcommand{\BigF}[2]{\ensuremath{#1 \left(#2\right)}}
\newcommand{\Wrap}[1]{\ensuremath{\left(#1\right)}}
\newcommand{\Q}[1]{\subsubsection*{Question #1}}


\begin{document}

\title{MAP433 - Projet 4}
\author{EL KHADIR Bachir, BIAD Anas}


\maketitle

Le code utilisé pour trouver les valeurs numériques et sa sortie sont disponibles en annexe.

\Q{1) - a}
Pour $mois \in \{ mai, juin \}$, on a:
$$\nabla_\vartheta M^{(mois)}(\hat{\vartheta}^{(mois)})
=  \nabla_\vartheta M^{(mois)}(\vartheta^{(mois)}) + H_e . (\hat{\vartheta}^{(mois)} -  \vartheta^{(mois)}) + \BigO{ \left\| \hat{\vartheta}^{(mois)}- \vartheta^{(mois)} \right\| ^2}  $$
où $H_e$ est la hessienne de $M^{(mois)}$ au point $\vartheta^{(mois)}$..

$M^{(mois)}$ atteint son minimum en $\hat{\vartheta}^{(mois)}$, donc: $\nabla_\vartheta M^{(mois)}(\hat{\vartheta}^{(mois)}) = 0$.
Par conséquent:


$$ \sqrt n (\hat{\vartheta}^{(mois)} -  \vartheta^{(mois)}) = -\sqrt n \, H_e ^{-1} \nabla_\vartheta M^{(mois)}(\vartheta^{(mois)})  + \BigO{ \sqrt n \, \left\|H_e ^{-1} (\hat{\vartheta}^{(mois)}- \vartheta^{(mois)}) \right\| ^2}    $$

Or
\begin{align*}
(\forall \vartheta \in \mathbb{R} ) \, \nabla_\vartheta M^{(mois)}(\vartheta) 
&= \frac{-2}{n} \sum_{i=1}^{k} \sum_{j=1}^{r} (Y_{i,j}^{(mois)} - f(x_i, \vartheta)) \nabla_\vartheta f(x_i, \vartheta) \\
&= \frac{-2}{n} \sum_{i=1}^{k} \sum_{j=1}^{r}  \varepsilon_{i,j} \, \nabla_\vartheta f(x_i, \vartheta) 
\end{align*}

$$ \mathbb{E} [\sqrt n \nabla_\vartheta M^{(mois)} (\vartheta^{(mois)}) ] = 0$$ 
car les $\varepsilon_{i,j}$ sont centrées.


\begin{align*}
\mathbb{V} [ \sqrt n \nabla_\vartheta M^{(mois)} (\vartheta^{(mois)} ] 
&= 4 \, \sum_{i=1}^{k} \sum_{j=1}^{r} \mathbb{E} [ \varepsilon_{i,j}^2 ] \, \nabla_\vartheta f(x_i, \vartheta^{(mois)}) \nabla_\vartheta f(x_i, \vartheta^{(mois)})^T 
& \text{ (car les $\varepsilon_{i,j}$ sont iid) } \\
&= \frac{4 \sigma^2}{n} \, \sum_{i=1}^{k} \sum_{j=1}^{r}  \nabla_\vartheta f(x_i, \vartheta^{(mois)}) \nabla_\vartheta f(x_i, \vartheta^{(mois)})^T  \\
&= \frac{4 \sigma^2}{k} \, \sum_{i=1}^{k} \nabla_\vartheta f(x_i, \vartheta^{(mois)}) \nabla_\vartheta f(x_i, \vartheta^{(mois)})^T  \\
& \to 4\sigma^2 \int^1_0 \nabla_\vartheta f(x, \vartheta^{(mois)}) \nabla_\vartheta f(x, \vartheta^{(mois)})^T \, \mathrm{d}x \\
&= 4\sigma^2 H(\vartheta^{(mois)} )
\end{align*}


Par TCL :
$$ \sqrt n \nabla_\vartheta M^{(mois)} (\vartheta^{(mois)} \overset{d}{\to} \mathcal{N}(0, 4 \sigma^2 H(\vartheta) )$$
On a aussi:
\begin{align*}
H_e &= \frac{-2}{n} \sum_{i, j} - \nabla_\vartheta f(x_i, \vartheta^{(mois)})\nabla_\vartheta f(x_i, \vartheta^{(mois)})^T + \varepsilon_{i,j} H(f) \\
&=  \frac{2}{n} \sum_{i, j} - \nabla_\vartheta f \nabla_\vartheta f ^T + \frac{-2}{n}  \sum_{i, j} \varepsilon_{i,j} H(f) 
\end{align*}
$ \frac{-2}{n}  \sum_{i, j} \varepsilon_{i,j} H(f) \longrightarrow 0$ par LGN, 
et $ \frac{2}{n} \sum_{i, j} - \nabla_\vartheta f \nabla_\vartheta f ^T \longrightarrow \BigF{H}{\vartheta^{(mois)}}$ 

d'où $$ H_e \overset{\mathbb{P}_{\vartheta} } \longrightarrow  2 \BigF{H}{\vartheta^{(mois)}} $$

On en déduit par Slutsky que:
$$ \sqrt n (\hat{\vartheta}^{(mois)} - \vartheta^{(mois)} ) \overset{ d } \longrightarrow \BigF{\mathcal{N}}{0, \sigma^2 \BigF{H}{\vartheta^{(mois)}}^{-1}}$$

Comme $\hat{\vartheta}^{(mai)}$ et $\hat{\vartheta}^{(juin)} $ sont indépendants, alors:
$$ \sqrt n (\hat{\vartheta}- \vartheta ) \overset{ d } \longrightarrow \BigF{\mathcal{N}}{0, \sigma^2 \BigF{H}{\vartheta}^{-1}}$$

où 
\begin{center}
\ensuremath{
\BigF{H}{\vartheta} = 
\begin{pmatrix}
 \BigF{H}{\vartheta^{(mai)}} & 0 \\
 0 & \BigF{H}{\vartheta^{(juin)}}
\end{pmatrix}
}
\end{center}

\Q{b)}
Posons $A = (\mathbb{I}_3 \,  0 \, -\mathbb{I}_3 \, 0) \in {\cal M}_{3, 8}(\mathbb{R})$ de rang $3$, alors $A \vartheta = 0$.

$\sqrt n (\hat{\vartheta}_n - \vartheta)  \overset d \longrightarrow \BigF{\mathcal{N}}{0, \sigma^2 \BigF{H}{\vartheta}^{-1}}$
Si $A \vartheta = 0$, alors par Slutsky: 
$$ \sqrt n A( \hat{\vartheta} - \vartheta ) \overset d \longrightarrow \BigF{\mathcal{N}}{0, \sigma^2 A \BigF{H}{\vartheta}^{-1}A^T}$$

\Q{c) }
On propose 
$$\sigma_n^2 = \frac{ \sum_{i,j} (\varepsilon_{i,j}^{(mai)})^2  + \sum_{i,j} (\varepsilon_{i,j}^{(juin)})^2}{2n}  \overset{ps} \longrightarrow \sigma^2$$ 
Par LGN 
parceque $\mathbb{V}[{\epsilon_{i,j}^{(mois)}}] = \sigma^2$.

Comme $\sqrt n (\hat{\vartheta} - \vartheta)  \overset d \longrightarrow \BigF{\mathcal{N}}{0, \sigma^2 \BigF{H}{\vartheta}^{-1}}$, alors
$\hat{\vartheta} - \vartheta  \overset {\mathbb{P}_{\vartheta}} \longrightarrow 0$.

Si $A\vartheta \neq 0$, alors 
$$(A\hat \vartheta_n)^T \hat V (\hat \vartheta_n) ^{-1} A \hat \vartheta _n 
\overset {\mathbb{P}_{\vartheta}} \longrightarrow 
(A\vartheta)^T \hat V ( \vartheta) ^{-1} A  \vartheta \neq 0$$

donc $$T_n \overset {\mathbb{P}_{\vartheta}} \longrightarrow + \infty$$
Si $A\vartheta = 0$ alors comme on a:
\begin{itemize}
	\item  
	$\sqrt n A( \hat{\vartheta} - \vartheta ) \overset d \longrightarrow \BigF{\mathcal{N}}{0, \sigma^2 V( \vartheta )}$
	\item
	$ \hat{V}(\hat \vartheta_n ) \overset { \mathbb{P}_{\vartheta}}{ \longrightarrow} V( \vartheta )$
	\item
	$\sigma_n^2 \overset { \mathbb{P}_{\vartheta}}{ \longrightarrow} \sigma^2$
\end{itemize}
Alors par Slutsky:
$$ T_n = \frac{1}{\hat \sigma_n^2 } ( \sqrt n A \hat \vartheta _n)^T \hat{V}(\hat \vartheta_n ) ( \sqrt n A \hat \vartheta _n)
\overset d \longrightarrow \frac{1}{\sigma^2} Z^T V(\vartheta) Z $$
où $Z \sim \BigF{\mathcal{N}}{0, \sigma^2 V( \vartheta )}$.

Or $V( \vartheta )$ est définie positive, en effet:
\begin{align*}
(\forall X \in \mathbb{R}^3 - \{0\}) \, X^T V( \vartheta ) X > 0 
& \Leftrightarrow  X^T A H(\vartheta) A^T X > 0 \\
& \Leftrightarrow  X^T A H(\vartheta) A^T X > 0 \\
& \Leftrightarrow  (A^T X)^T H(\vartheta) A^T X > 0 & \text{(car $H(\vartheta) > 0$ et $A^T X \neq 0$)} \\
\end{align*}

Donc:
$$ \frac{1}{\sigma^2} Z^T V(\vartheta) Z = \left\| \frac{V(\vartheta)^{-\frac{1}{2}}}{\sigma} Z \right\|^2 \sim \chi^2(3) $$

\Q{d)}
On propose le test suivant:

$$
\varphi(\hat \vartheta _n) = \left\{
    \begin{array}{ll}
        1 & T_n > \beta \\
        0 & \mbox{sinon}
    \end{array}
\right.
$$
Sous \textbf{$H_0$}, si $\vartheta$ et tel que $A \vartheta  = 0$:
\begin{align*}
\mathbb{P}_\vartheta(\varphi = 1) 
&= \mathbb{P}_\vartheta(T_n > \beta ) \\
&  \to 1 - \mathbb{P}_\vartheta( \chi^2(3) \leq \beta ) \\
&= 5\%
\end{align*}
Où $\beta = 7.81$ est  le quantile de $\chi2(3)$ à $95\%$.

\Q{2 a) }
$\hat \vartheta_n = 
\bigl(\begin{smallmatrix}
0.042 & 0.059 \\
1.94 & 1.91 \\
2.56 & 2.84 \\
3.47 & 3.25 \\
\end{smallmatrix} \bigr)$,
$T_n = 2.1 < \beta$ donc $\varphi = 0$. \textbf{On ne peut rien dire}.

\Q{2 b)}
On trouve $\hat \rho = 0.6$

\Q{3 a)}
$I_B(\alpha = 5\%) = \left[ 0.48, 0.75 \right] $

\Q{3 b)}
Posons $g(x) = 10^x$, de dérivé $g'(x) = ln(10) g(x)$.

\begin{align*}
\sqrt n (\hat \rho - \rho) 
&= \sqrt n \left( 10^{\tilde \vartheta^{(juin)}_4 -  \tilde \vartheta^{(mai)}_4} - 10^{ \vartheta^{(juin)}_4 - \vartheta^{(mai)}_4} \right) \\
&= \sqrt n \left( \BigF{g}{ \tilde \vartheta^{(juin)}_4 -  \tilde \vartheta^{(mai)}_4 } - \BigF{g}{ \vartheta^{(juin)}_4 - \vartheta^{(mai)}_4} \right)
\end{align*}
Par le théorème des accroissement finis, il existe $\tilde \eta \in \left( \tilde \vartheta^{(juin)}_4 -  \tilde \vartheta^{(mai)}_4, \vartheta^{(juin)}_4 - \vartheta^{(mai)}_4 \right)$ tel que
$$\sqrt n (\hat \rho - \rho) 
= g'( \tilde \eta ) \left(  \tilde \vartheta^{(juin)}_4 - \vartheta^{(juin)}_4 +  \tilde \vartheta^{(mai)}_4  - \vartheta^{(mai)}_4 \right)
$$
Sous $H_0$ on a:
$$ \sqrt n ( \tilde{\vartheta} - \vartheta ) \overset d \longrightarrow \BigF{\mathcal{N}}{0, \sigma^2  \BigF{H}{\vartheta}^{-1}}$$
donc par projection sur la quatrième coordonnée:
$$ \tilde \vartheta^{(juin)}_4 - \vartheta^{(juin)}_4 \overset d \longrightarrow \BigF{\mathcal{N}}{0, \sigma^2 \nu_1}$$
et
$$ \tilde \vartheta^{(mai)}_4 - \vartheta^{(mai)}_4 \overset d \longrightarrow \BigF{\mathcal{N}}{0, \sigma^2 \nu_2}$$
où $\nu_1$ (resp. $\nu_2$ ) est la composante $(4,4)$ de la matrice $H(\vartheta^{(juin)})^{-1}$ (resp. $H(\vartheta^{(mai)})^{-1}$)

Par indépendance:
$$  \tilde \vartheta^{(juin)}_4 - \vartheta^{(juin)}_4 
+ \tilde \vartheta^{(mai)}_4 - \vartheta^{(mai)}_4 
 \overset d \longrightarrow \BigF{\mathcal{N}}{0, \sigma^2 (\nu_1 + \nu_2) }$$

Et comme:
$$  g'( \tilde \eta ) \overset {\mathbb{P}_{\vartheta}} \longrightarrow g'( \vartheta^{(juin)}_4 -  \vartheta^{(mai)}_4)
$$
Alors par Slutsky:
$$ \sqrt n (\hat \rho - \rho)  \overset d \longrightarrow \BigF{\mathcal{N}}{0, s^2 }$$
avec $s = g'( \vartheta^{(juin)}_4 -  \vartheta^{(mai)}_4) \, \sigma \, \sqrt{\nu_1 + \nu_2}$.

On estime $s$ par  $\hat s = g'( \tilde \vartheta^{(juin)}_4 -  \tilde \vartheta^{(mai)}_4)  \, \sigma_n \, \sqrt{\hat \nu_1 + \hat \nu_2}$, avec
$\hat \nu_1$ (resp. $\hat \nu_2$ ) est la composante $(4,4)$ de la matrice $H(\hat \vartheta^{(juin)})^{-1}$ (resp. $H(\hat \vartheta^{(mai)})^{-1}$).

On a $\hat s \overset {\mathbb{P}_{\vartheta}} \longrightarrow s$.

\Q{ b) }
\begin{align*}
1 - \alpha &= \mathbb{P}\left(\hat Z ^ b \in \left[G^*(\alpha/2), G^*(1-\alpha/2)\right]\right) \\
&= \mathbb{P}\left(G^*(\alpha/2) < \sqrt \frac{n}{\hat s_b}(\hat \rho_b - \rho) < G^*(1-\alpha/2)\right) \\
&= \mathbb{P}\left(\hat \rho_b - \sqrt \frac{\hat s_b}{n}G^*(1-\alpha/2) <  \rho < \hat \rho_b - \sqrt \frac{\hat s_b}{n}G^*(\alpha/2)\right)
\end{align*}

Donc $ I'_B(\alpha) = [ \rho_b - \sqrt \frac{\hat s_b}{n}G^*(1-\alpha/2), \hat \rho_b - \sqrt \frac{\hat s_b}{n}G^*(\alpha/2)]$ est un intervalle de confiance (non asymptotique) de niveau $\alpha$.

\subsection*{Code du programme:}
\inputminted{r}{script.R}

\subsection*{Sortie du programme:}
\inputminted{r}{sortie.R}

\end{document}


