\documentclass{article}

\usepackage[utf8]{inputenc}
\usepackage{amsmath}
\usepackage{amsfonts}
\usepackage[final]{pdfpages} 
\usepackage{cases}
\usepackage{soul}

\usepackage[top=2.25cm, bottom=2.25cm, left=2cm, right=2cm]{geometry}
\newcommand{\g}[2]{g_{#1}^{#2}}


\begin{document}

\section{First time of reach for a sum of a Markov chain}

\subsection{Defining the problem}

Let $(X_{n})_{n\geq0}$ be a markov chain such as :
\begin{itemize}
\item $\forall \, n\geq0 \quad X_{n} \in \{-1,1\} $ 
\item $\forall \, i\geq0 \quad\ \mathbb{P}( X_{i+1}=1 | X_{i}=1 ) = \mathbb{P}( X_{i+1}=-1 | X_{i}=-1 ) = p = 1-q$ \footnote{We assume that $p \ne 0$ and $p \ne 1$. Otherwise, the problem is trivial.}
\end{itemize}
Let $(S_{n})_{n\geq0}$ be a sequence such as :
$$ S_{n} = \sum_{k=1}^{n}X_{k} $$
Let $T$ be the first time when $S_{n}$ reaches $-1$, i.e  :
$$ T= \inf \{ n\geq0 , S_{n}=-1   \} $$
Let us define the  probability-generating function of T such as :
$$ g_{s}^{x}(z) = \mathbb{E}(z^{T} | S_{0}=s, X_{0}=x) \quad \forall \ x \in \{-1,1\} \ , \ s \in \mathbb{N} \ , \ z \in [0,1]  $$

\subsection{The probability-generating function dynamic}
Given the transition matrix of $(X_{n})_{n\geq0}$, we can infer that for all $x \in \{-1,1\}$ , $s \in \mathbb{N}$ and $z \in ]0,1]$ \footnote{For clarity, we omit to note the dependence of G in z.} :

\begin{numcases}
		\strut 
       	\g{s}{1} = z(p\g{s+1}{1}+ q\g{s-1}{-1})\\
       	\g{s}{-1} = z(q\g{s+1}{1}+ p\g{s-1}{-1}) 
\end{numcases}
Thus, for $z > 0$ \footnote{The case of $z=0$ is trivial since $ \g{s}{x}(0)=0$ } :
\begin{align}
		\g{s+1}{1} = \frac{1}{z p} \g{s}{1} - \frac{q}{p} \g{s-1}{-1} \\
		\g{s+1}{1} = \frac{1}{ zq} \g{s}{-1} - \frac{p}{q} \g{s-1}{-1}
\end{align}
Which implies that :
\begin{equation}
		\g{s}{-1} =\frac{q}{p} \g{s}{1}+\frac{z (p^{2}-q^{2})}{p} \g{s-1}{-1}
\end{equation}
Leading to :
\begin{equation}
		\g{s+1}{-1} =  (\frac{1}{z p}   + z(2 - \frac{1}{p})) \g{s}{-1}  -  \g{s-1}{-1} 
\end{equation}
Consequently : 
$$
G_{s+1} = 
\begin{pmatrix} 
\g{s+1}{1} \\ \g{s+1}{-1} \\ \g{s}{1} \\ \g{s}{-1} 
\end{pmatrix}
= 
\begin{pmatrix} 
\frac{1}{z p} & 0 &  0 &  -\frac{q}{p} \\
0 & \frac{1}{ zp}   + z(2 - \frac{1}{p}) & 0 & -1 \\
1 & 0 &  0 & 0\\
0 & 1 &  0 & 0
\end{pmatrix}
.
\begin{pmatrix} 
\g{s}{1} \\ \g{s}{-1} \\ \g{s-1}{1} \\ \g{s-1}{-1} 
\end{pmatrix} \\
$$
By noting :

$$
A = A(z) = 
\begin{pmatrix} 
\frac{1}{z p} & 0 &  0 &  -\frac{q}{p} \\
0 & \frac{1}{z p}   + z(2 - \frac{1}{p}) & 0 & -1 \\
1 & 0 &  0 & 0\\
0 & 1 &  0 & 0
\end{pmatrix}
$$
and by imposing a Dirichlet boundary condition we finally get the following second order system :  

\begin{numcases}
		\strut 
       	 G_{s+1}(z) = A(z). G_{s}(z)\\
       	\g{-1}{1} = \g{-1}{-1} = 0\\
       	\lim_{s \to +\infty} \g{s}{-1} = \lim_{s \to +\infty} \g{s}{1} = 0
\end{numcases}

\subsection{Expression of the characteristic polynomial of the recursion matrix A }
For clarity we define the following notations :
$$
a= \frac{1}{z p}
\quad,\quad
b= z(2-\frac{1}{p})
\quad and \quad
c= -\frac{q}{p}
$$
such that :
$$
A = 
\begin{pmatrix} 
a & 0 &  0 &  c \\
0 & a+b & 0 & -1 \\
1 & 0 &  0 & 0\\
0 & 1 &  0 & 0
\end{pmatrix}
$$
The characteristic polynomial of the matrix A can be thus written :
\begin{equation}
\chi_{A}(X) = det(A-XI_{4}) = X(X-a)P(X)
\end{equation}
where : 
\begin{equation}
P_{z}(X) = X^{2}-(a+b)X+1 
\end{equation} 
In order to determine all the eigenvalues of A we need to study the polynomial P.

\subsection{Studying the polynomial P}
For recall :
$$ 
P_{z}(X) = X^{2}-(a+b)X+1 
$$
Thus, the discriminant is given by :
\begin{equation}
\Delta_{P}(z) = (a+b)^{2}-4 = \frac{[z-1][z+1][(2p-1)z+1][(2p-1)z-1]}{(z p)^{2}}
\end{equation}


\begin{flushleft}
\textbf{Case :}  $p=\frac{1}{2}$
\medbreak
In this case, $a=\frac{2}{z}$ and $b=0$. It implies that :
$$
\Delta_{P}(z) = (\frac{2}{z})^{2}-4 = 4 (\frac{1}{z^2}-1) \geq 0
$$
Thereby :
$$ 
P_{z}(X) = \left(X-\frac{1}{z}-\sqrt{\frac{1}{z^2}-1}\right)\left(X-\frac{1}{z}+\sqrt{\frac{1}{z^2}-1}\right)
$$
\end{flushleft}


\begin{flushleft}
\textbf{Case :}  $p \in ]0,1[$ 
\medbreak
In this case :
\begin{equation}
\Delta_{P}(z) = (a+b)^{2}-4 = 
\frac{[z^{2}-1][(2p-1)^{2}z^{2}-1]}{(z p)^{2}}
\end{equation}
Since $z \in ]0,1]$ and $p \in ]0,1[$, we get : 
$$
\Delta_{P}(z) \geq 0 \quad  \forall z\in ]0,1]
$$
Thus, $P_{z}$ can be written :
$$
	P_{z}(X) = (X-x_{1}(z))(X-x_{2}(z))
$$
where :
\begin{numcases}
		\strut 
       	x_{1}(z) = \frac{a+b-\sqrt{(a+b)^{2}-4}}{2}\\
       	x_{2}(z) = \frac{a+b+\sqrt{(a+b)^{2}-4}}{2} 
\end{numcases}
\end{flushleft}

\begin{flushleft}
\textbf{Remarks :} 
\medbreak
\begin{itemize}
\item $ x_{1}(z) = x_{2}(z) \Leftrightarrow \Delta_{P}(z) = 0 \Leftrightarrow z = 1 $ \footnote{$z \in ]0,1] $}
\item The case where $ p=\frac{1}{2} $ is highlighted because it is a typical case.
\end{itemize}
\end{flushleft}

Finally, all information required for the reduction of the matrix A is now avaible.

\subsection{Reduction of the matrix A}
\begin{flushleft}
\textbf{Case :}  $z \in ]0,1[$ 
\medbreak
In this case,  $ x_{1}(z) \ne x_{2}(z) $ and the characteristic polynomial of the matrix A can be thus written :
\begin{equation}
\chi_{A}(X) = det(A-XI_{4}) = X(X-a)(X-x_{1}(z))(X-x_{2}(z))
\end{equation}
We can easily check that \footnote{At least, when $ p \ne 1 $} :
\begin{itemize}
\item $a \ne 0$
\item $x_{1}(z) \notin \{0,a\}$ and $x_{2}(z) \notin \{0,a\}$
\end{itemize}

On the other hand, the spectrum of the matrix A is given by: 
\begin{equation}
Sp_{\mathbb{R}}(A) =\{0,a,x_{1}(z),x_{2}(z)\}
\end{equation}
With the associated eigenspaces \footnote{For clarity, we omit to note the dependence of $x_{1}$ and $x_{2}$ in z.} :
$$
 E_{A}(0) = vect
 \begin{pmatrix}
 0\\
 0\\
 1\\
 0
 \end{pmatrix}
 \quad
 E_{A}(a) = vect
 \begin{pmatrix} 
 a\\
 0\\
 1\\
 0
 \end{pmatrix}
 \quad
 E_{A}(x_{1}) = vect
 \begin{pmatrix} 
 \frac{c}{x_{1}-a}\\
 x_{1}\\
 \frac{c}{x_{1}(x_{1}-a)}\\
 1
 \end{pmatrix}
 \quad
 E_{A}(x_{2}) = vect
 \begin{pmatrix} 
 \frac{c}{x_{2}-a}\\
 x_{2}\\
 \frac{c}{x_{2}(x_{2}-a)}\\
 1
 \end{pmatrix}
$$
Leading to :
\begin{equation}
A(z) = P(z).D(z).P(z)^{-1}
\end{equation}

Where :
\begin{equation}
P(z) = 
\begin{pmatrix} 
0 & a &  \frac{c}{x_{1}-a} &  \frac{c}{x_{2}-a} \\
0 & 0 & x_{1} & x_{2} \\
1 & 1 &  \frac{c}{x_{1}(x_{1}-a)} &  \frac{c}{x_{2}(x_{2}-a)}\\
0 & 0 &  1 & 1
\end{pmatrix}
\quad \quad
D(z) = 
\begin{pmatrix} 
0 & 0 & 0 &  0 \\
0 & a & 0 & 0\\
0 & 0 & x_{1}(z) &  0\\
0 & 0 &  0&  x_{2}(z)
\end{pmatrix}
\end{equation}

\end{flushleft}

\begin{flushleft}
\textbf{Case :}  $z = 1$ 
\medbreak
In this case,  $ a= \frac{1}{p} $ and $ b= 2- \frac{1}{p}$. Which implies that :
\begin{equation}
x_{1}(z) = x_{2}(z) = 1
\end{equation}
Thus, the spectrum of the matrix A is given by: 
\begin{equation}
Sp_{\mathbb{R}}(A) =\{0,\frac{1}{p},1\}
\end{equation}

With the associated eigenspaces  :
$$
 E_{A}(0) = vect
 \begin{pmatrix} 
 0\\
 0\\
 1\\
 0
 \end{pmatrix}
 \quad
 E_{A}(\frac{1}{p}) = vect
 \begin{pmatrix} 
 \frac{1}{p}\\
 0\\
 1\\
 0
 \end{pmatrix}
 \quad
 E_{A}(1) = vect
 \begin{pmatrix} 
 1\\
 1\\
 1\\
 1
 \end{pmatrix} 
$$
Thereby, the matrix A can be written in a Jordan form :
\begin{equation}
A = Q.T.Q^{-1}
\end{equation}

Where :
\begin{equation}
Q = 
\begin{pmatrix} 
0 & \frac{p}{q} &  -1 &  \frac{p}{q} \\
0 & 0 & -1 & 0 \\
-2q & \frac{p^{2}}{q} &  -1 &  2-\frac{p}{q}\\
0 & 0 &  -1 & 1
\end{pmatrix}
\quad \quad
T = 
\begin{pmatrix} 
0 & 0 & 0 &  0 \\
0 & \frac{1}{p} & 0 & 0\\
0 & 0 & 1 &  1\\
0 & 0 &  0&  1
\end{pmatrix}
\end{equation}
\end{flushleft}

\begin{flushleft}
\textbf{Remarks :} 
\medbreak
By a simple induction, we can easily prove that :
\begin{equation}
\forall \, s\in \mathbb{N} \quad
D(z)^{s} = 
\begin{pmatrix} 
0 & 0 & 0 &  0 \\
0 & a^{s} & 0 & 0\\
0 & 0 & x_{1}(z)^{s} &  0\\
0 & 0 &  0&  x_{2}(z)^{s}
\end{pmatrix}
\end{equation}
and
\begin{equation}
\forall \, s\in \mathbb{N} \quad
T^{s} = 
\begin{pmatrix} 
0 & 0 & 0 &  0 \\
0 & \frac{1}{p^{s}} & 0 & 0\\
0 & 0 & 1 &  s\\
0 & 0 &  0&  1
\end{pmatrix}
\end{equation}
\end{flushleft}




\end{document}
