%%% Preamble
\documentclass[paper=a4, fontsize=11pt]{scrartcl}
\usepackage[utf8]{inputenc}  
\usepackage[T1]{fontenc}
\usepackage{fourier}
\usepackage[english]{babel}															% English language/hyphenation
\usepackage[protrusion=true,expansion=true]{microtype}	
\usepackage{amsmath,amsfonts,amsthm} % Math packages
\usepackage[pdftex]{graphicx}	
\usepackage{url}

%%% Custom sectioning
\usepackage{sectsty}
\allsectionsfont{\centering \normalfont\scshape}

%%% Custom headers/footers (fancyhdr package)
\usepackage{fancyhdr}
\pagestyle{fancyplain}
\fancyhead{}											% No page header
\fancyfoot[L]{}											% Empty 
\fancyfoot[C]{}											% Empty
\fancyfoot[R]{\thepage}									% Pagenumbering
\renewcommand{\headrulewidth}{0pt}			% Remove header underlines
\renewcommand{\footrulewidth}{0pt}				% Remove footer underlines
\setlength{\headheight}{13.6pt}


%%% Equation and float numbering
\numberwithin{equation}{section}		% Equationnumbering: section.eq#
\numberwithin{figure}{section}			% Figurenumbering: section.fig#
\numberwithin{table}{section}				% Tablenumbering: section.tab#


%%% Maketitle metadata
\newcommand{\horrule}[1]{\rule{\linewidth}{#1}} 	% Horizontal rule

\title{
		%\vspace{-1in} 	
		\usefont{OT1}{bch}{b}{n}
		\normalfont \normalsize \textsc{Ecole Polytechnique} \\ [25pt]
		\horrule{0.5pt} \\[0.4cm]
		\huge Modèles des Taux d'interêt \\
		\horrule{2pt} \\[0.5cm]
}
\author{
		\normalfont \normalsize
                Bachir EL KHADIR\\[-3pt] \normalsize
                \today	
}
\date{}

\newtheorem{theorem}{Theorem}

%%% Begin document
\begin{document}
\maketitle

\newpage

%%% Local Variables:
%%% mode: latex
%%% TeX-master: "main.tex"
%%% End:

\section{Remerciments}

Ecole Polytechnique
Maitre de stage
Bla bla bla
\newpage

%%% Local Variables:
%%% mode: latex
%%% TeX-master: "main"
%%% End:

\newpage

\subsection*{Résumé}
  
  Ce rapport présente quelques outils utilisés lors du pricing de dérivés de taux. 
  Il est particulièrement centré sur le modèle de Hull White qui est conduit par deux facteurs (appelé aussi G2++).  J'adopterai le plan suivant dans la rédaction de ce rapport:
  \begin{itemize}
  \item 
    La première partie introduit le marché de taux par un certain nombre de définitions qui lui sont propres, ainsi que les produits financiers principaux utilisés pour la calibration des modèles.
  \item 
    Dans la deuxième partie je présenterai quelques modèles existant, et je discuterai de leurs limites et de leurs implémentions pratique.
  
  \item
    La troisième partie présente les résultats concrets que j'ai obtenus dans le cas du modèle gaussien à deux facteurs, ainsi que la partie calibration de modèle.
  \end{itemize}
  
\subsection*{Abstract} 
This report presents some tools used in the pricing of intereset rates derivatives. It's focused on two factors Hull-White model (called G2++).  
  \begin{itemize}
  \item
    The first part introduces some definitions relative to the interest rates market, as well as some derivative used later in the calibration.
  \item
    In the second part I will present some existing models, their limitations and possible implementation.

  \item
    The last part presents the results I obtained using the G2++ model and the calibration process.
  \end{itemize}




\newpage
\chapter*{Introduction}

J'ai fait mon stage a JP Morgan équipe exotic rates. La plus grosse partie du travail consistait à trouver le bon modèle mathématique pour pricer un produit et le calibrer au marché.

L'intérêt que porte les banques aux produits exotiques a considérablement augmenté au cours des dernières années. Plusieurs modèles ont été développé pour refléter au mieux le comportement des marchés financiers.

Le département au sein duquel j'ai effectué mon stage s'occupait des produits basé sur les taux d'intérêt. Dans ce domaine, une propriété appréciable dans un modèle est le fait qu'il possède un nombre de paramètres suffisant pour être calibré parfaitement aux prix observables dans la réalité, sans pour autant ``overfitter'' l'échantillon disponible, surtout que dernièrement, depuis la crise des subprimes, le marché des taux a connu des changement radicaux que les modèles traditionnels n'arrive pas décrire. La possibilité d'observer des taux négatifs en est un exemple. 

Un produit en particulier n'a cessé de  gagner en notoriété: les cancelables spread options. Ma mission de stage était de comprendre les modèles existant et leurs implémentations qui permettent de pricer (entre autres) ce type de produits, comprendre leurs limites, et essayer de trouver des améliorations.

Ce stage est à forte composante informatique, une attention particulière a été accordée aux détails d'implémentation et optimisation du code.
 
\newpage

La notion de taux d'interêt nous est familière et fait parti de notre vie de tous les jours.

Zéro Coupon

\section{Quelques définitions}
\begin{theorem}
$Z(t, T, S)$ est le montant qu'il faut investir dans un instrument risque-neutre au temps $T$ pour obtenir 1euro au temps S.
\end{theorem}
\begin{theorem}
$r$ le taux d'intêret instantné est défini comme étant
\end{theorem}
\begin{theorem}
La courbe de rendement ou la yield curve $L(t, T)$
\end{theorem}
\begin{theorem}
Le taux d'interet cumulé
\end{theorem}

\section{Produits financier d'interêts}
Swap:
Un swap est un contract entre deux parties qui s'engagent à échanger des flux financier pendant une durée et à une fréquence détérminées. la plupart du temps, ces flux sont détérminé comme étant l'intêret sur un notionnel K. 
$$ N \sum D_t^{T_i} \tau_i (L(T_{i-1}) - K) $$

Caps/floor:
Un cap/floor peut être vu comme un call/put européenne sur un swap
$$ N \sum D_t^{T_i} \tau_i (L(T_{i-1}) - K)^+ $$

Swaption
$$ N (\sum D_t^{T_i} \tau_i (L(T_{i-1}) - K))^+ $$
\section{Les différents modèles des taux d'interêts}

Certains produits financiers dépendent en grande partie de la courbe de rendement. Nous avons vu précédemment que la donnée du taux instantané $r_t$ permet de caractériser complètement cette courbe.

Il est donc important que la dynamque de $r_t$ soit à la fois riche pour pouvoir décrire la courbe de rendement observée dans le marché, et suffisemment simple pour que le temps nécessaire pour le calcul ne soit trop long.

Nous pouvons considérer que la courbe de rendement varie dans un espace vectoriel de dimension infinie. Toute tentative de la caracteriser par un nombre fini de paramètre rééls est donc vouée à l'échec.


\section{Le modèle à deux facteurs}
\subsection{Motivation}



Montrons dans un premier temps pourquoi un modèle à un seul facteur n'est pas suffisant pour pricer certains produits disponible sur les marchés. En effet, considérons un produit $E$ dont le payoff  dépent  du spread entre un taux d'intêret cumulé entre $0$ et $T_1$ pour le premier et $0$ et $T_2$ pour le second.

La dynamique de $r_t$ dans le modèle de Vasicek est donné par

$$r_t = k(\theta - r_t)  \mathrm{d}t  + \sigma \mathrm{d}W_t$$
La formule analytique du bond zéro coupon est donc
$$P(t, T) = A(t, T) exp(-B(t, T) r_t)$$
En particulier le taux d'intêret cumulé est une donné par une transformation affine du taux instantané:
$$R(t, T) = \frac{ln P(t, T)}{T-t} =: a(t, T) + b(t, T) r_t$$
Le payoff du produit $E$ est donc fonction de la distribution jointe de $R(0, T_1)$ et $R(0, T_2)$. Sauf que:
$$Cor(R(0, T_1), R(0, T_2) = 1$$
On en déduit qu'un choc à $r_t$ agit de la même manière sur toutes les courbes
On observe sur les marchés que les taux à différentes maturités ne sont pas corrélés. Si on regarde le taux 2Y et 10Y

<Courbe 10Y> 

<Courbe 2Y>

Clairement un tel modèle ne capture pas ce comportement.

Dans cette section nous considérons un modèle où le taux d'intêret instantanté est donné par une somme de deux facteurs gaussiens centrés et corrélés. Pour fitter la yield courbe, on rajoute une fonction détérministe.

\begin{align*}
  \rm{d}X &= \\
  \rm{d}Y &= \\
  \rm{d} r &= X + Y + \phi(t)
\end{align*}

$$P(t, T) = E[...]$$

Le processus est markovien

Les cap floors ademettent une formule analytiques, comme pour les options vanilles : formule de blackscholes
Dans la pratique, les banques tradent beaucoup des produits exotiques. Ce n'est pas le cas pour ces produits => arbre de pricing
\subsection{Approximation de la solution par un arbre binomial}

\begin{itemize}
\item Discrétisation
\item Construction de l'abre
\end{itemize}

\subsection{Formule analytique - Intégration}

\subsection{Performance}
L'arbre est long mais beaucoup plus puissant
Imperfections de l'arbre:
\begin{itemize}
\item  bornee
\item  discretisation
\item  probabilite negative
\end{itemize}
Monte Carlo
Limitation
Closed Form

Le temps d'execution:
Arbre
Closed form

\section{Application: calibration et pricing}
Le modèle à 2F permet de caputrer le hump de la courbe de rendement
Le nombre de paramètre est fini (5) => pas de overfitting




\end{document}

%%% Local Variables:
%%% mode: latex
%%% TeX-master: t
%%% End:
